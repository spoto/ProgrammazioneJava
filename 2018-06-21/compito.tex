\documentclass[12pt]{article}
\usepackage[pdftex]{graphicx}
\usepackage{amsfonts}
\usepackage[italian]{babel}
\usepackage{graphicx}
\usepackage{color}

\definecolor{grey}{rgb}{0.3,0.3,0.3}

\usepackage{listings, framed}
\lstset{
  language=Java,
  showstringspaces=false,
  columns=flexible,
  basicstyle={\small\ttfamily},
  frame=none,
  numbers=none,
  keywordstyle=\bfseries\color{grey},
  commentstyle=\itshape\color{red},
  identifierstyle=\color{black},
  stringstyle=\color{blue},
  numberstyle={\ttfamily},
%  breaklines=true,
  breakatwhitespace=true,
  tabsize=3,
  escapechar=|
}

%****************enlarge layout
\textheight     243.5mm
\topmargin      -20.0mm
\textwidth      480pt
\hoffset        -80pt
%*****************theorems and such
\newcounter{esnu}
\newenvironment{esercizio}{\medskip \noindent {\bf Esercizio\addtocounter{esnu}{1} \arabic{esnu}}}{}
\pagestyle{empty}
\newcommand{\liff}{\mathrel{\leftrightarrow}}   % Logical IFF Symbol
\newcommand{\metaiff}{\Longleftrightarrow}      %iff in metatheory

\begin{document}

%\begin{tabular}{llclcr}
% \hspace{-35pt} &{\bf COGNOME:} & \hspace{100pt}        &{\bf NOME:}    & \hspace{100pt}        &{\bf MATRICOLA:}%\hspace{35pt} \\
%\hline
%\end{tabular}
\begin{center} {\bf Esame di Programmazione II, 21 giugno 2018}\end{center}
%\`

Si consideri un'interfaccia che rappresenta un numero non negativo,
in una qualsiasi base di numerazione:
%
\begin{lstlisting}
public interface Number {
  int getValue();
}
\end{lstlisting}

\begin{esercizio}~\textbf{[8 punti]}
Si completi la seguente implementazione astratta
di un \texttt{Number}, che fornisce le funzionalit\`a comuni a tutti i numeri,
cio\`e il controllo sulla non negativit\`a del valore, l'accesso al valore, la traduzione
in stringa e i metodi per il test di uguaglianza:
%
\begin{lstlisting}
public abstract class AbstractNumber implements Number {
  private final int value;

  protected AbstractNumber(int value) {
    // lancia una IllegalArgumentException se value e' negativo,
    // altrimenti inizializza il campo value
    ...
  }

  // restituisce il valore di questo numero
  @Override public final int getValue() { ... }

  // restituice la base di numerazione di questo numero
  protected abstract int getBase();

  // restituisce il carattere che rappresenta la cifra "digit" nella base di numerazione
  // di questo numero. Sara' sempre vero che 0 <= digit < getBase()
  protected abstract char getCharForDigit(int digit);

  // restituisce una stringa che rappresenta il numero nella sua base di numerazione
  @Override public String toString() { ... }

  @Override public final boolean equals(Object other) {
    // due numeri sono uguali se e solo se hanno lo stesso valore
    ...
  }

  @Override public final int hashCode() {
    // deve essere compatibile con equals() e non banale
    ...
  }
}
\end{lstlisting}
\end{esercizio}

\begin{esercizio}~\textbf{[6 punti]}
  Si scrivano le sottoclassi concrete \texttt{DecimalNumber},
  \texttt{BinaryNumber}, \texttt{OctalNumber} e \texttt{HexNumber} di
  \texttt{AbstractNumber}, che rappresentano, rispettivamente, un numero
  in base $10$, $2$, $8$ e $16$. Queste classi si instanziano con
  il loro costruttore, a cui viene passato il valore del numero.
  Non si ridefinisca, in queste quattro sottoclassi,
  il metodo \texttt{toString()}: quello ereditato da \texttt{AbstractNumber}
  dovr\`a funzionare per tutte queste sottoclassi, traducendo il valore del numero
  nella giusta base di numerazione.
\end{esercizio}

\begin{esercizio}~\textbf{[5 punti]}
  Nella codifica binaria con parit\`a, un numero binario viene esteso con un'ulteriore cifra
  binaria di controllo,
  in modo da rendere pari il numero totale di cifre \texttt{1}. Se quindi il numero
  binario aveva una quantit\`a pari di \texttt{1}, si aggiunger\`a una cifra di controllo
  \texttt{0}. Se invece il numero binario aveva una quantit\`a dispari di \texttt{1},
  si aggiunger\`a una cifra di controllo \texttt{1}. Questa modifica riduce il rischio
  di trasmissione di dati corrotti, permettendo di implementare un rudimentale sistema di
  rilevazione dell'errore. Si implementi una sottoclasse concreta
  \texttt{BinaryNumberWithParity} di \texttt{BinaryNumber}, ridefinendo il metodo
  \texttt{toString()} in modo da aggiungere in fondo la cifra di controllo opportuna.
\end{esercizio}

\begin{esercizio}~\textbf{[5 punti]}
  Nella codifica in base 58, si utilizzano 58 cifre diverse, scelte fra i numeri arabi e le lettere
  inglesi maiuscole e minuscole. Si evitano i caratteri \texttt{0OIl}, che potrebbero
  essere confusi a video, perch\'e graficamente simili. Si implementi una sottoclasse
  concreta \texttt{Base58Number} di \texttt{AbstractNumber}, in modo da implementare
  questa numerazione in base 58. Le 58 cifre sono quindi
  \texttt{123456789ABCDEFGHJKLMNPQRSTUVWXYZabcdefghijkmnopqrstuvwxyz}.
  Non si ridefinisca il metodo \texttt{toString()} ereditato da \texttt{AbstractNumber}.
\end{esercizio}

\begin{esercizio}~\textbf{[5 punti]}
  Si completi la seguente classe di prova:

\begin{lstlisting}
public class Main {
  public static void main(String[] args) {
    try (Scanner keyboard = new Scanner(System.in)) {
      System.out.print("Inserisci un numero non negativo: ");
      int n = keyboard.nextInt();

      List<Number> numbersAsList = new ArrayList<>();
      // si aggiunga a numbersAsList il numero n, 6 volte:
      // prima in base 10, poi in base 2, poi 2 con parita', poi 8, poi 16 e infine 58
      ...
      System.out.println("lista: " + numbersAsList);

      Set<Number> numbersAsSet = ...
      // si copino tutti gli elementi della lista numbersAsList dentro l'insieme numbersAsSet
      ...
      System.out.println("insieme: " + numbersAsSet);
    }
    // se si verifica una IllegalArgumentException, la si intercetti, stampando su video
    // un messaggio che chiarisca che ci si aspettava un numero non negativo
    ...
  }
}
\end{lstlisting}
\end{esercizio}
%
Se tutto \`e  corretto, l'esecuzione della classe \texttt{Main} dovr\`a stampare ad esempio:
%
\begin{lstlisting}
Inserisci un numero non negativo: 1234567
lista: [1234567, 100101101011010000111, 1001011010110100001111, 4553207, 12D687, 7Kze]
insieme: ...
\end{lstlisting}  
%
\begin{esercizio}~\textbf{[2 punti]}
  Cosa verr\`a stampato alla linea precedente, al posto dei puntini di sospensione?
\end{esercizio}

\begin{center}
\textbf{Si possono definire altri campi, metodi, costruttori e classi, ma solo \texttt{private}.}
\end{center}

\end{document}
