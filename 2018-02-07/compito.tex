\documentclass{article}[10pt]
\usepackage[pdftex]{graphicx}
\usepackage{amsfonts}
\usepackage[italian]{babel}
\usepackage{graphicx}
\usepackage{color}

\definecolor{grey}{rgb}{0.3,0.3,0.3}

\usepackage{listings, framed}
\lstset{
  language=Java,
  showstringspaces=false,
  columns=flexible,
  basicstyle={\small\ttfamily},
  frame=none,
  numbers=none,
  keywordstyle=\bfseries\color{grey},
  commentstyle=\itshape\color{red},
  identifierstyle=\color{black},
  stringstyle=\color{blue},
  numberstyle={\ttfamily},
%  breaklines=true,
  breakatwhitespace=true,
  tabsize=3,
  escapechar=|
}

%****************enlarge layout
\textheight     243.5mm
\topmargin      -20.0mm
\textwidth      480pt
\hoffset        -80pt
%*****************theorems and such
\newcounter{esnu}
\newenvironment{esercizio}{\medskip \noindent {\bf Esercizio\addtocounter{esnu}{1} \arabic{esnu}}}{}
\pagestyle{empty}
\newcommand{\liff}{\mathrel{\leftrightarrow}}   % Logical IFF Symbol
\newcommand{\metaiff}{\Longleftrightarrow}      %iff in metatheory

\begin{document}

%\begin{tabular}{llclcr}
% \hspace{-35pt} &{\bf COGNOME:} & \hspace{100pt}        &{\bf NOME:}    & \hspace{100pt}        &{\bf MATRICOLA:}%\hspace{35pt} \\
%\hline
%\end{tabular}
\begin{center} {\bf Esame di Programmazione II, 7 febbraio 2018}\end{center}
%\`

Si consideri un'implementazione di un sistema di prenotazione di stanze \emph{HairBnB}.
Le stanze possono essere aggiunte al sistema di prenotazione, specificando il numero
massimo di ospiti e il prezzo richiesto al giorno. I clienti possono scrivere delle
recensioni (\emph{review}) per le stanze, indicando una valutazione da una a cinque stelle.
Il sistema di prenotazione permette di selezionare tutte le stanze che soddisfano dei
requisiti di ricerca (numero ospiti e prezzo), eventualmente ordinate rispetto a un qualche
criterio di ordinamento.

\textbf{Nota: } in questo compito non si ridefiniscano i metodi \texttt{equals()} e
\texttt{hashCode()}, anche se si aggiungessero oggetti in \texttt{HashSet}: vanno bene
quelli ereditati da \texttt{Object}.

\begin{esercizio}~\textbf{[5 punti]}
Si completi la seguente classe, che rappresenta una recensione di un utente:
%
\begin{lstlisting}
public class Review {
  public enum Stars { ONE, TWO, THREE, FOUR, FIVE }
  ...

  // costruisce una recensione con autore, testo e valutazione da 1 a 5 stelle
  public Review(String author, String text, Stars stars) { ... }

  // ritorna una stringa di asterischi, lunga da 1 a 5, sulla base delle stelle della recensione
  private String stars() { ... }

  // ritorna l'autore seguito dal risultato di stars() seguito dal testo della recensione
  @Override public String toString() { ... }

  // ritorna il numero di stelle della recensione, da 1 a 5
  public int howManyStars() { ... }
}
\end{lstlisting}
%
\end{esercizio}

\begin{esercizio}~\textbf{[6 punti]}
Si completi la seguente classe, che rappresenta una stanza:
%
\begin{lstlisting}
public class Room {
  ...

  // costruisce una stanza con una descrizione, un numero massimo di occupanti e un prezzo giornaliero
  public Room(String description, int people, int pricePerDay) { .... }

  // aggiunge la review a quelle di questa stanza
  public void addReview(Review review) { ... }

  // restituisce la descrizione della stanza, seguita dal numero medio di stelle, seguito
  // dal numero massimo di persone, seguito dal prezzo giornaliero richiesto,
  // seguito dal toString() di ciascuna recensione ricevuta da questa stanza
  @Override public String toString() { ... }

  // ritorna il numero massimo di persone che possono occupare la stanza
  public int getPeople() { ... }

  // ritorna il prezzo giornaliero richiesto
  public int getPriceForDay() { ... }

  // ritorna la media delle stelle delle recensioni ricevute; ritorna 0.0 in caso di nessuna recensione
  public double averageStars() { ... }
}
\end{lstlisting}
\end{esercizio}

\begin{esercizio}~\textbf{[6 punti]}
Si completi la seguente classe, che rappresenta il sistema di prenotazione di
un insieme di stanze. \`E possibile aggiungere stanze al sistema ed \`e possibile
ottenere l'insieme delle stanze compatibili con il criterio di selezione prescelto
(numero minimo di persone e prezzo giornaliero massimo richiesto):
%
\begin{lstlisting}
public class HairBnB {
  ...

  // aggiunge a questo sistema le stanze fornite come argomenti
  public void addRooms(Room... rooms) { ... }

  // restituisce l'insieme delle stanze che abbiano spazio per almeno minPeople
  // e che costino al massimo maxPrice al giorno; questo metodo deve lanciare
  // una NoRoomAvailableException se nessuna stanza soddisfacesse tali richieste
  public Set<Room> availableFor(int minPeople, int maxPrice) { ... }
}
\end{lstlisting}
\end{esercizio}

\begin{esercizio}~\textbf{[1 punto]}
Si definisca l'eccezione \texttt{NoRoomAvailableException} per l'esercizio~3.
\end{esercizio}

\begin{esercizio}~\textbf{[7 punti]}
%
Si aggiunga il seguente metodo alla classe \texttt{HairBnB} dell'esercizio~3:
%
\begin{lstlisting}
public SortedSet<Room> sortedAvailableFor(int minPeople, int maxPrice, Comparator<Room> comparator) {
  ...
}
\end{lstlisting}
%
che si comporta esattamente come \texttt{availableFor()}, incluso il lancio dell'eccezione,
ma si differenzia solo per il fatto che restituisce un insieme ordinato di stanze.
L'ordinamento \`e specificato dal parametro \texttt{comparator}, che implementa l'interfaccia della
libreria standard:
%
\begin{lstlisting}
public interface Comparator<T> {
  // restituisce:
  // un numero negativo se o1 viene prima di o2;
  // un numero positivo se o2 viene prima di o1;
  // 0 se o1 e o2 sono uguali
  int compare(T o1, T o2);
}
\end{lstlisting}
%
\textbf{Suggerimento: } normalmente la classe di libreria \texttt{TreeSet<T>} realizza
un insieme ordinato di \texttt{T} e tale tipo generico deve implementare \texttt{Comparable<T>}.
\`E per\`o possibile definire dei \texttt{TreeSet<T>} anche per tipi \texttt{T} che non
implementino \texttt{Comparable<T>}, purch\'e il criterio di confronto venga fornito al
momento della costruzione dell'insieme, tramite un \texttt{Comparator<T>}, cos\`{\i}: \texttt{new TreeSet<T>(comparator)}.
\end{esercizio}

\begin{esercizio}~\textbf{[7 punti]}
Si completi la seguente classe di prova, inizializzando le variabili
\texttt{queryByPrice} e \texttt{queryByStars}:
%
\begin{lstlisting}
public class Main {
  public static void main(String[] args) {
    Room room1 = new Room("Room with view in Borgo Roma", 2, 30);
    Room room2 = new Room("Nice flat in Chievo", 4, 70);
    Room room3 = new Room("Historic room in Verona", 1, 80);

    room1.addReview(new Review("Albert Einstein",
      "I loved the room, large and with view on the hospital", Review.Stars.FOUR));
    room1.addReview(new Review("Liz Taylor", "Ugly place, full of hair", Review.Stars.ONE));
    room2.addReview(new Review("Vasco Rossi", "Really lovely", Review.Stars.FIVE));

    HairBnB bnb = new HairBnB();
    bnb.addRooms(room1, room2, room3);

    // queryByPrice devono essere le stanze per almeno una persona, che costano al massimo 70 al giorno,
    // ordinate per prezzo crescente (cioe' prima quelle che costano meno)
    SortedSet<Room> queryByPrice = ...

    // come queryByPrice, ma ordinate per valutazione decrescente (cioe' prima quelle con valutazione maggiore)
    SortedSet<Room> queryByStars = ...
  }
}
\end{lstlisting}
\end{esercizio}

\begin{center}
\textbf{\`E possibile definire campi, metodi, costruttori e classi aggiuntive, ma solo \texttt{private}.}
\end{center}

\end{document}
