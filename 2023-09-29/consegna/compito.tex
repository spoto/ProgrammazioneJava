\documentclass[12pt]{article}
\usepackage[pdftex]{graphicx}
\usepackage{amsfonts}
\usepackage[italian]{babel}
\usepackage{graphicx}
\usepackage{color}
\usepackage{multirow,bigdelim}
\usepackage{relsize}
\usepackage{fdsymbol}
\usepackage{mdframed}

\definecolor{grey}{rgb}{0.3,0.3,0.3}
\definecolor{verylightgray}{rgb}{.97,.97,.97}
\definecolor{lightred}{rgb}{1,.70,.70}

\usepackage{listings, framed}
\lstset{
  language=Java,
  showstringspaces=false,
  columns=flexible,
  basicstyle={\small\ttfamily},
  frame=none,
  numbers=none,
  keywordstyle=\bfseries\color{grey},
  commentstyle=\itshape\color{red},
  identifierstyle=\color{black},
  stringstyle=\color{blue},
  numberstyle={\ttfamily},
%  breaklines=true,
  breakatwhitespace=true,
  tabsize=3,
  escapechar=|
}

\mdfsetup{font=\scriptsize}

\def\codesize{\smaller}
\def\<#1>{\codeid{#1}}
\newcommand{\codeid}[1]{\ifmmode{\mbox{\codesize\ttfamily{#1}}}\else{\codesize\ttfamily #1}\fi}

%****************enlarge layout
\textheight     243.5mm
\topmargin      -20.0mm
\textwidth      500pt
\hoffset        -80pt
%*****************theorems and such
\newcounter{esnu}
\newenvironment{esercizio}{\medskip \noindent {\bf Esercizio\addtocounter{esnu}{1} \arabic{esnu}}}{}
\pagestyle{empty}
\newcommand{\liff}{\mathrel{\leftrightarrow}}   % Logical IFF Symbol
\newcommand{\metaiff}{\Longleftrightarrow}      %iff in metatheory

\begin{document}

\begin{center}
  \textbf{Esame di Programmazione II, 29 settembre 2023}\\
  (si consegnino i file \texttt{.java})
\end{center}

\emph{
Si crei un progetto Eclipse e
il package \<it.univr.instructions>. Si copino al suo interno
le classi del compito.
Non si modifichino le dichiarazioni dei metodi e delle classi. Si possono definire altri campi,
metodi, costruttori e classi, ma devono essere \<private>.
La soluzione che verr\`a consegnata dovr\`a compilare,
altrimenti non verr\`a corretta.}

\vspace*{2ex}

Un \emph{macchina a stack} \`e un calcolatore che esegue una sequenza di istruzioni
(\emph{programma}), il cui
effetto \`e di modificare uno stack di interi. Tipicamente, l'esecuzione di ciascuna
istruzione aggiunge e/o toglie uno o pi\`u elementi dalla cima dello stack.
Le istruzioni sono le seguenti:
%
\begin{itemize}
\item \texttt{push(c)}: aggiunge la costante intera $c$ in cima allo stack;
\item \texttt{pop}: rimuove dallo stack l'intero che si trova in cima;
  se lo stack fosse vuoto, blocca l'esecuzione con un errore;
\item \texttt{add/sub/mul/div}: rimuove dallo stack i due interi $i_1$ ed $i_2$ che si trovano in cima
  (cio\`e la cima $i_2$ dello stack e l'elemento subito sotto $i_1$) e aggiunge al loro posto
  un intero, risultato della addizione/sottrazione/moltiplicazione/divisione (rispettivamente)
  di $i_1$ per $i_2$; se lo stack avesse meno di due elementi, blocca l'esecuzione
  con un errore; nel caso della \texttt{div}, se $i_2$ fosse zero blocca l'esecuzione
  con un errore (non si pu\`o dividere per zero);
\item \texttt{repeat(c, ins)}: esegue $c$ volte l'istruzione \texttt{ins}; si avr\`a sempre
  $c\ge 0$. Questa istruzione pu\`o generare un errore nel caso in cui \texttt{ins} generasse
  un errore.
\end{itemize}

Per esempio, l'esecuzione del programma
%
\begin{center}
  \texttt{push(5) push(13) push(17) add}
\end{center}
%
parte dallo stack vuoto, aggiunge $5$ in cima allo stack,
poi aggiunge $13$ in cima allo stack, poi aggiunge $17$ in cima
allo stack e infine toglie $13$ e $17$ dallo stack e aggiunge $30$ al loro posto.
Alla fine lo stack conterr\`a $5$ e, in cima, $30$. Il risultato dell'esecuzione
\`e proprio il numero ($30$ nel nostro esempio) che si trova infine in cima allo stack.
Un altro esempio \`e
%
\begin{center}
  \texttt{push(13) add}
\end{center}
%
In questo caso l'esecuzione del programma fallisce perch\'e l'istruzione \texttt{add}
viene eseguita in uno stack che ha solo un elemento, $13$, mentre la \texttt{add} richiede
almeno due elementi nello stack per potere essere eseguita correttamente.

\vspace*{2ex}\textbf{Esercizio 1 ($2$ punti).}
Si implementi l'eccezione controllata \texttt{IllegalProgramException}, con un unico costruttore,
che riceve il messaggio (stringa) che descrive perch\'e una esecuzione \`e fallita.

\vspace*{2ex}\textbf{Esercizio 2 ($7$ punti).}
L'iterfaccia \texttt{Instruction} descrive le istruzioni: hanno un metodo
che esegue l'istruzione a partire da un certo stack e un altro metodo per ottenere una
rappresentazione stringa dell'istruzione, comodo per poterla stampare:

\begin{lstlisting}[language=Java]
public interface Instruction {
  void execute(List<Integer> stack) throws IllegalProgramException;
  String toString();
}
\end{lstlisting}
%
Si definiscano le classi \texttt{PUSH}, \texttt{ADD}, \texttt{SUB},
\texttt{MUL} e \texttt{DIV}, che implementano \texttt{Instruction}.
\textbf{Suggerimento: } si guardi la classe \texttt{POP}, gi\`a fatta
e completa, e ci si ispiri ad essa.

\vspace*{2ex}\textbf{Esercizio 3 ($7$ punti).}
Si implementi la classe \texttt{REPEAT}, che implementa \texttt{Instruction},
il cui unico costruttore riceve $c$ e \texttt{ins}. Se $c$
fosse negativo, deve lanciare una
\texttt{IllegalArgumentException}.

\vspace*{2ex}\textbf{Esercizio 4 ($8$ punti).}
Una macchina a stack \`e descritta dall'interfaccia
\texttt{Machine}, il cui unico metodo restituisce il risultato dell'esecuzione
di un programma:
%
\begin{lstlisting}[language=Java]
public interface Machine {
  int getResult();
}
\end{lstlisting}
%
Si completi l'implementazione \texttt{SimpleMachine} di \texttt{Machine},
nell'unico \texttt{TODO} indicato nel codice.

\vspace*{2ex}\textbf{Esercizio 5 ($6$ punti).}
Si definisca una sottoclasse \texttt{PrintingMachine} di
\texttt{SimpleMachine} che si differenzia solo per un dettaglio:
per ciascuna istruzione, che viene eseguita, stampa l'istruzione, seguita
dallo stack risultante dopo la sua esecuzione.

\vspace*{2ex}
\hrule
\vspace*{2ex}

Se tutto \`e corretto, l'esecuzione del \texttt{Main} (gi\`a completo, da non
modificare) stamper\`a:

\begin{mdframed}[backgroundcolor=lightred]
  {\small\begin{verbatim}
     * * * Eseguo [push(5), push(13), push(17), add] * * *

Simple machine:
result = 30

Printing machine:
push(5): [5]
push(13): [5, 13]
push(17): [5, 13, 17]
add: [5, 30]
result = 30

 * * * Eseguo [repeat(5, push(13)), repeat(4, add)] * * *

Simple machine:
result = 65

Printing machine:
repeat(5, push(13)): [13, 13, 13, 13, 13]
repeat(4, add): [65]
result = 65

 * * * Eseguo [repeat(5, push(13)), repeat(5, add)] * * *

Simple machine:
Errore: Operandi insufficienti per un'operazione binaria

Printing machine:
repeat(5, push(13)): [13, 13, 13, 13, 13]
repeat(5, add): Errore: Operandi insufficienti per un'operazione binaria
\end{verbatim}}
\end{mdframed}

\end{document}
