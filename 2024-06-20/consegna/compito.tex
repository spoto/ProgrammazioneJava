\documentclass[12pt]{article}
\usepackage[pdftex]{graphicx}
\usepackage{amsfonts}
\usepackage[italian]{babel}
\usepackage{graphicx}
\usepackage{color}
\usepackage{multirow,bigdelim}
\usepackage{relsize}
\usepackage{fdsymbol}
\usepackage{mdframed}

\definecolor{grey}{rgb}{0.3,0.3,0.3}
\definecolor{verylightgray}{rgb}{.97,.97,.97}
\definecolor{lightred}{rgb}{1,.70,.70}

\usepackage{listings, framed}
\lstset{
  language=Java,
  showstringspaces=false,
  columns=flexible,
  basicstyle={\small\ttfamily},
  frame=none,
  numbers=none,
  keywordstyle=\bfseries\color{grey},
  commentstyle=\itshape\color{red},
  identifierstyle=\color{black},
  stringstyle=\color{blue},
  numberstyle={\ttfamily},
%  breaklines=true,
  breakatwhitespace=true,
  tabsize=3,
  escapechar=|
}

\mdfsetup{font=\scriptsize}

\def\codesize{\smaller}
\def\<#1>{\codeid{#1}}
\newcommand{\codeid}[1]{\ifmmode{\mbox{\codesize\ttfamily{#1}}}\else{\codesize\ttfamily #1}\fi}

%****************enlarge layout
\textheight     243.5mm
\topmargin      -20.0mm
\textwidth      500pt
\hoffset        -80pt
%*****************theorems and such
\newcounter{esnu}
\newenvironment{esercizio}{\medskip \noindent {\bf Esercizio\addtocounter{esnu}{1} \arabic{esnu}}}{}
\pagestyle{empty}
\newcommand{\liff}{\mathrel{\leftrightarrow}}   % Logical IFF Symbol
\newcommand{\metaiff}{\Longleftrightarrow}      %iff in metatheory

\begin{document}

\begin{center}
  \textbf{Esame di Programmazione II, 20 giugno 2024}\\
  (si consegni \texttt{AbstractComponent.java}, \texttt{FileComponent.java} e \texttt{DirectoryComponent.java})
\end{center}

\emph{
Si crei un progetto Eclipse e
il package \<it.univr.file>. Si copino al suo interno
le classi del compito.
Non si modifichino le dichiarazioni dei metodi e delle classi. Si possono definire altri campi,
metodi o costruttori non richiesti dal compito, ma devono essere \<private>. Si possono definire altre classi,
che in tal caso vanno consegnate.
La soluzione che verr\`a consegnata dovr\`a compilare,
altrimenti non verr\`a corretta.}

\vspace*{2ex}

Un file system permette di memorizzare \emph{componenti}, che possono essere
file oppure directory (dette anche \emph{folder} o \emph{raccoglitori}).
Un file \`e un documento con una dimensione in byte. Una directory \`e un contenitore di zero o pi\`u
sottocomponenti (i suoi \emph{figli}), che a loro volta
possono essere sia file che directory. Si noti che sia i file che le directory hanno un nome,
ma solo le directory hanno dei figli. Sia file che directory hanno una dimensione in byte. Nel caso di un file,
questa dimensione \`e specificata dal file stesso (il suo numero di byte) mentre nel caso
di una directory la dimensione \`e data da $100$ byte pi\`u la dimensione dei sui figli, ricorsivamente.

Per esempio il seguente programma crea un albero di file e directory radicato in \<root>, lo stampa,\
ne stampa la dimensione, stampa la lista dei file raggiungibili a partire da \<root> e infine stampa
qual \`e il percorso, a partire da \<root>, che porta ai file \<dog.gif> e \<Pluto.c>:

\begin{lstlisting}[language=Java]
package it.univr.file;

import java.io.FileNotFoundException;

public class MainFiles {

  public static void main(String[] args) throws FileNotFoundException {
    Component f1 = new FileComponent("cat1.jpg", 34590);
    Component f2 = new FileComponent("dog.gif", 12422);
    Component f3 = new FileComponent("cat2.jpg", 52402);
    Component images = new DirectoryComponent("images", f1, f2, f3);
    Component music = new DirectoryComponent("music"); // directory vuota
    Component f4 = new FileComponent("Pippo.java", 3255);
    Component f5 = new FileComponent("Paperino.c", 44341);
    Component work = new DirectoryComponent("work", f4, f5);
    Component f6 = new FileComponent("passwords.txt", 3233);
    Component root = new DirectoryComponent("root", work, images, f6, music);
    System.out.println(root);
    System.out.println();
    System.out.println("total size: " + root.size() + " bytes");
    System.out.println("files: " + root.getFiles());
    System.out.println("dog.gif si trova come " + root.find("dog.gif"));
    System.out.println("Pluto.c si trova come " + root.find("Pluto.c")); // eccezione!
  }
}
\end{lstlisting}


\newpage
\noindent
La sua esecuzione dovr\`a stampare:

\begin{mdframed}[backgroundcolor=lightred]
  {\small\begin{verbatim}
root/
  images/
    cat1.jpg
    cat2.jpg
    dog.gif
  music/
  passwords.txt
  work/
    Paperino.c
    Pippo.java

total size: 150643 bytes
files: [cat1.jpg, cat2.jpg, dog.gif, passwords.txt, Paperino.c, Pippo.java]
dog.gif si trova come root/images/dog.gif
Exception in thread "main" java.io.FileNotFoundException: Pluto.c
\end{verbatim}}
\end{mdframed}
%
Si noti che la stampa della directory avviene in ordine alfabetico crescente per
nome della componente.
Si noti inoltre che \<Pluto.c> non esiste, quindi la sua ricerca finisce in eccezione.

\vspace*{2ex}\textbf{Esercizio 1 ($7$ punti).}
L'interfaccia \<Component.java> \`e gi\`a scritta e completa. Non va modificata, ma si legga
con attenzione i commenti dei suoi metodi, perch\'e dovranno essere implementati dalle sottoclassi.
Si completi quindi la sua sottoclasse astratta \<AbstractComponent.java> che implementa le parti comuni
a file e directory, cio\`e soltanto
\<getName()> e \<toString()>: tutti gli altri metodi rimarranno astratti e verrano implementati
nelle due sottoclassi concrete nei prossimi esercizi.

\vspace*{2ex}\textbf{Esercizio 2 ($10$ punti).}
Si completi la sottoclasse concreta \<FileComponent.java> della classe astratta
\<AbstractComponent.java>.
Dal momento che si tratta di una classe concreta, essa dovr\`a implementare
tutti i metodi che ancora sono astratti a questo livello della gerarchia delle classi.

\vspace*{2ex}\textbf{Esercizio 3 ($14$ punti).}
Si completi la sottoclasse concreta \<DirectoryComponent.java> della
classe astratta \<AbstractComponent.java>.
Dal momento che si tratta di una classe concreta, essa dovr\`a implementare
tutti i metodi che ancora sono astratti a questo livello della gerarchia delle classi.

\end{document}
