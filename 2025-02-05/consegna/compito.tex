\documentclass[12pt]{article}
\usepackage[pdftex]{graphicx}
\usepackage{amsfonts}
\usepackage[italian]{babel}
\usepackage{graphicx}
\usepackage{color}
\usepackage{multirow,bigdelim}
\usepackage{relsize}
\usepackage{fdsymbol}
\usepackage{mdframed}

\definecolor{grey}{rgb}{0.3,0.3,0.3}
\definecolor{verylightgray}{rgb}{.97,.97,.97}
\definecolor{lightred}{rgb}{1,.70,.70}

\usepackage{listings, framed}
\lstset{
  language=Java,
  showstringspaces=false,
  columns=flexible,
  basicstyle={\small\ttfamily},
  frame=none,
  numbers=none,
  keywordstyle=\bfseries\color{grey},
  commentstyle=\itshape\color{red},
  identifierstyle=\color{black},
  stringstyle=\color{blue},
  numberstyle={\ttfamily},
%  breaklines=true,
  breakatwhitespace=true,
  tabsize=3,
  escapechar=|
}

\mdfsetup{font=\scriptsize}

\def\codesize{\smaller}
\def\<#1>{\codeid{#1}}
\newcommand{\codeid}[1]{\ifmmode{\mbox{\codesize\ttfamily{#1}}}\else{\codesize\ttfamily #1}\fi}

%****************enlarge layout
\textheight     243.5mm
\topmargin      -20.0mm
\textwidth      500pt
\hoffset        -80pt
%*****************theorems and such
\newcounter{esnu}
\newenvironment{esercizio}{\medskip \noindent {\bf Esercizio\addtocounter{esnu}{1} \arabic{esnu}}}{}
\pagestyle{empty}
\newcommand{\liff}{\mathrel{\leftrightarrow}}   % Logical IFF Symbol
\newcommand{\metaiff}{\Longleftrightarrow}      %iff in metatheory

\begin{document}

\begin{center}
  \textbf{Esame di Programmazione II, 5 febbraio 2025}\\
  (si consegni \<AbstractTime.java>, \<ItalianTime.java>, \<AmericanTime.java> e \<Interval.java>)
\end{center}

\emph{
Si crei un progetto Eclipse e
il package \<it.univr.time>. Si copino al suo interno
le classi del compito.
Non si modifichino le dichiarazioni dei metodi e delle classi. Si possono definire altri campi,
metodi o costruttori non richiesti dal compito, ma devono essere \<private>. Si possono definire altre classi,
che in tal caso vanno consegnate.
La soluzione che verr\`a consegnata dovr\`a compilare,
altrimenti non verr\`a corretta.}

\vspace*{2ex}

L'interfaccia \<Time> rappresenta un istante di tempo (ore, minuti, secondi). Ne esistono
due implementazioni, \<ItalianTime> e \<AmericanTime>, che si differenziano per il modo in cui
vengono rappresentate dal metodo \<toString>. Il codice comune fra queste due implementazioni
\`e in \<AbstractTime>. Per costruire una \<ItalianTime> o una \<AmericanTime> si indicano
i secondi passati dall'inizio del giorno (cio\`e dalle 00:00:00 di quel giorno). I metodi
di \<Time> permettono di ottenere l'istante successivo con il metodo \<next> e un intervallo
di istanti successivi con il metodo \<interval>. Per capire meglio, si consideri il
seguente \<Main.java>, gi\`a scritto e da non modificare:
%
\begin{lstlisting}[language=Java]
public class Main {
  public static void main(String[] args) {
    Time t1 = new ItalianTime(12 * 3600 + 22 * 60 + 59); // le 12:22:59
    Time t2 = new AmericanTime(12 * 3600 + 22 * 60 + 59); // le 12:22:59
    Time t3 = new ItalianTime(23 * 3600 + 59 * 60 + 59); // le 23:59:59
    Time t4 = new AmericanTime(23 * 3600 + 59 * 60 + 59); // le 23:59:59
    System.out.println("t1 = " + t1); System.out.println("t2 = " + t2);
    System.out.println("t3 = " + t3); System.out.println("t4 = " + t4);
    System.out.println("t3.next() = " + t3.next());
    System.out.println("t4.next() = " + t4.next());

    System.out.println("10 secondi da t3 in poi:");
    for (Time t: t3.interval(10))
      System.out.println(t);

    System.out.println("10 secondi da t4 in poi:");
    for (Time t: t4.interval(10))
      System.out.println(t);
  }
}
\end{lstlisting}
%
che dovrebbe stampare:
%
\begin{mdframed}[backgroundcolor=lightred]
  {\small\begin{verbatim}
t1 = 12:22:59
t2 = 12:22:59pm
t3 = 23:59:59
t4 = 11:59:59pm
t3.next() = 00:00:00
t4.next() = 12:00:00am
10 secondi da t3 in poi:
23:59:59
00:00:00
00:00:01
00:00:02
00:00:03
00:00:04
00:00:05
00:00:06
00:00:07
00:00:08
10 secondi da t4 in poi:
11:59:59pm
12:00:00am
12:00:01am
12:00:02am
12:00:03am
12:00:04am
12:00:05am
12:00:06am
12:00:07am
12:00:08am
\end{verbatim}}
\end{mdframed}
%
Si noti la differenza di stampa fra le ore italiane e quelle americane, che usano
am per la mattina e pm per il pomeriggio:
%
\begin{center}
  \begin{tabular}{|c|c|}
    \hline
    italiana & americana \\\hline\hline
    00 & 12am \\\hline
    01 & 01am \\\hline
    02 & 02am \\\hline
    11 & 11am \\\hline
    12 & 12pm \\\hline
    13 & 01pm \\\hline
    14 & 02pm \\\hline
    15 & 03pm \\\hline
    23 & 11pm \\\hline
  \end{tabular}
\end{center}

\vspace*{2ex}\textbf{Esercizio 1 ($7$ punti).}
Si completi la classe \<AbstractTime.java>, che contiene il codice comune alle due
rappresentazioni di un istante. Si faccia riferimento ai commenti dei metodi da completare.

\vspace*{2ex}\textbf{Esercizio 2 ($11$ punti).}
Si completino le sottoclassi \<ItalianTime.java> e \<AmericanTime.java>, che si differenziano
solo per i metodi \<next> e \<toString>. \textbf{Suggerimento: } nella conversione da
ora italiana (00--23) a ora americana (am/pm) vi conviene gestire a parte i casi
00 e 12; inoltre pu\`o esservi utile il formato di stampa \<\%02d> di \<String.format>,
che rappresenta un \<int> su due cifre, riempiendo con \<0> la cifra a sinistra se l'intero
avesse solo una cifra.

\vspace*{2ex}\textbf{Esercizio 3 ($13$ punti).}
Si completi la classe \<Interval.java> che rappresenta un intervallo di istanti consecutivi.
Si noti che si tratta di un \<Iterable$\langle$Time$\rangle$>, quindi il suo metodo
\<iterator()> deve restituire un iteratore che ad ogni chiamata di \<next> restituisce
il prossimo \<Time> dell'intervallo. Si veda l'esempio di utilizzo nel \<main>, con i due
cicli foreach.

\end{document}
