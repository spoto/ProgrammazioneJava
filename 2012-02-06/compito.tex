\documentclass{article}[10pt]
\usepackage[pdftex]{graphicx}
\usepackage{amsfonts}
\usepackage[italian]{babel}
%****************enlarge layout
\textheight     243.5mm
\topmargin      -20.0mm
\textwidth      480pt
\hoffset        -80pt
%*****************theorems and such
\newcounter{esnu}
\newenvironment{esercizio}{\medskip \noindent {\bf Esercizio\addtocounter{esnu}{1} \arabic{esnu}}}{}
\pagestyle{empty}
\newcommand{\liff}{\mathrel{\leftrightarrow}}   % Logical IFF Symbol
\newcommand{\metaiff}{\Longleftrightarrow}      %iff in metatheory

\begin{document}

\begin{tabular}{llclcr}
 \hspace{-35pt} &{\bf COGNOME:} & \hspace{100pt}        &{\bf NOME:}    & \hspace{100pt}        &{\bf MATRICOLA:}\hspace{35pt} \\
\hline
\end{tabular}
\begin{center} {\bf Esame di Programmazione II, 6 febbraio 2012}\end{center}

\begin{esercizio}
\textbf{[7 punti]} Si scriva una classe \texttt{Television.java} che implementa un \emph{televisore},
sintonizzato su un canale. Tale classe deve avere un costruttore pubblico senza argomenti
che crea un televisore sintonizzato sul canale numero 1.
Inoltre deve avere i seguenti metodi pubblici:
%
\begin{itemize}
\item \texttt{getChannel}, che restituisce il numero del canale su cui il televisore
      \`e sintonizzato;
\item \texttt{setChannel}, che riceve come argomento il numero intero di un canale
      e sintonizza il televisore su tale canale;
\item \texttt{toString}, che restituisce una stringa che descrive il televisore, del
      tipo:
\begin{verbatim}
******************************
*        canale cinto        *
******************************
\end{verbatim}
      In particolare, il canale 1 si chiama \texttt{rave uno}, il canale 2
      \texttt{rave due}, il canale 3 \texttt{rave tre}, il canale 4
      \texttt{tele squatto} e il canale 5 \texttt{tele cinto}. Tutti gli altri
      canali si chiamano \texttt{canale sconosciuto}. La larghezza della
      stringa ritornata deve essere la stessa per tutti i nomi dei canali
      (nell'esempio, 30 caratteri).
\end{itemize}
%
Ogni altro metodo, costruttore o campo deve essere dichiarato \texttt{private}.
%
\end{esercizio}

\begin{esercizio}
\textbf{[11 punti]} Si definisca l'interfaccia \texttt{Command.java} che ha un unico metodo, chiamato
\texttt{execute}, senza argomenti e che ritorna \texttt{void}.
Si definisca un'implementazione
\texttt{SetChannelCommand.java} di \texttt{Command.java} il cui metodo \texttt{execute}
sintonizza un televisore su
un canale indicato con un numero intero. Entrambe queste informazioni devono in qualche
modo essere dentro un \texttt{SetChannelCommand}!
Si definisca un'implementazione
\texttt{EmptyCommand.java} di \texttt{Command.java} il cui metodo \texttt{execute}
non fa nulla.

Si definisca quindi una classe \texttt{Controller.java} che implementa un \emph{telecomando}
di un televisore, con solo 7 pulsanti a disposizione, identificati con $0\ldots 6$.
Un telecomando si crea con un costruttore
pubblico con un solo argomento: il televisore a cui il telecomando \`e connesso.
L'effetto \`e di costruire un telecomando in cui la pressione del tasto
%
\begin{enumerate}
\item[0.] sintonizza il televisore su \texttt{rave uno};
\item sintonizza il televisore su \texttt{rave due};
\item sintonizza il televisore su \texttt{rave tre};
\item sintonizza il televisore sul canale di numero intero successivo a quello attuale;
\item non fa nulla
\item non fa nulla
\item non fa nulla
\end{enumerate}
%
La classe \texttt{Controller.java} deve avere anche i metodi pubblici:
%
\begin{itemize}
\item \texttt{pressKey}, che preme il pulsante il cui numero \`e passato come
      argomento (tra 0 e 6, non controllate);
\item \texttt{programKey}, che riceve come argomento il numero intero di un pulsante
      (tra 0 e 6, non controllate) e un comando, e fa s\`{\i} che da quel momento
      quel pulsante eseguir\`a quel comando;
\item \texttt{undo}, che annulla l'effetto dell'ultimo pulsante premuto. Per esempio,
      se l'ultimo pulsante premuto ha sintonizzato il televisore dal canale 2 al canale 4,
      questo metodo lo risintonizza sul canale 2. \`E possibile eseguire pi\`u volte di seguito
      il metodo \texttt{undo}, tornando sempre pi\`u indietro nel tempo fino al canale
      iniziale su cui era sintonizzato il televisore all'inizio. Il metodo dovr\`a ritornare
      \texttt{true} se e solo se \`e stato possibile annullare l'ultimo comando, mentre
      \texttt{false} indicher\`a che non c'\`e pi\`u nessun comando da annullare.
\end{itemize}
%
Ogni altro metodo, costruttore o campo deve essere dichiarato \texttt{private}.

\textbf{Suggerimento}: per l'\texttt{undo} pu\`o essere utile gestire uno stack. A tal fine
usate gli stack gi\`a disponibili in Java: \texttt{java.util.Stack<E>} con un costruttore
senza argomenti e metodi
%
\begin{itemize}
\item \texttt{void push(E element)}
\item \texttt{E pop()}
\end{itemize}
%
\end{esercizio}
%
\begin{esercizio}
%
\textbf{[4 punti]} Si scriva una classe \texttt{Test.java} con un metodo \texttt{main} che effettua le seguenti
operazioni in sequenza:
%
\begin{enumerate}
\item crea un televisore e lo stampa
\item lo sincronizza sul canale 4 e lo stampa
\item crea un telecomando per quel televisore
\item preme il tasto 3 del telecomando e stampa il televisore
\item preme il tasto 6 del telecomando e stampa il televisore
\item preme il tasto 0 del telecomando e stampa il televisore
\item riprogramma il tasto 6 del telecomando in modo che sintonizzi il televisore
      sul canale di numero precedente a quello attuale
\item preme il tasto 6 del telecomando e stampa il televisore
\item esegue l'\texttt{undo} finch\'e c'\`e qualche pressione di tasto da annullare, stampando
      ogni volta il televisore.
\end{enumerate}
%
L'effetto del \texttt{main} dovrebbe essere di stampare qualcosa del tipo:
%
{\small\begin{verbatim}
******************************
*          rave uno          *
******************************
******************************
*       canale squatto       *
******************************
premo il tasto 3
******************************
*        canale cinto        *
******************************
premo il tasto 6
******************************
*        canale cinto        *
******************************
premo il tasto 0
******************************
*          rave uno          *
******************************
riprogrammo il tasto 6
premo il tasto 6
******************************
*     canale sconosciuto     *
******************************
undo
******************************
*          rave uno          *
******************************
undo
******************************
*        canale cinto        *
******************************
undo
******************************
*        canale cinto        *
******************************
undo
******************************
*       canale squatto       *
******************************
\end{verbatim}}
%
\end{esercizio}
%
\end{document}
