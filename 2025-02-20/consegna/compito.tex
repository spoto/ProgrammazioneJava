\documentclass[12pt]{article}
\usepackage[pdftex]{graphicx}
\usepackage{amsfonts}
\usepackage[italian]{babel}
\usepackage{graphicx}
\usepackage{color}
\usepackage{multirow,bigdelim}
\usepackage{relsize}
\usepackage{fdsymbol}
\usepackage{mdframed}

\definecolor{grey}{rgb}{0.3,0.3,0.3}
\definecolor{verylightgray}{rgb}{.97,.97,.97}
\definecolor{lightred}{rgb}{1,.70,.70}

\usepackage{listings, framed}
\lstset{
  language=Java,
  showstringspaces=false,
  columns=flexible,
  basicstyle={\small\ttfamily},
  frame=none,
  numbers=none,
  keywordstyle=\bfseries\color{grey},
  commentstyle=\itshape\color{red},
  identifierstyle=\color{black},
  stringstyle=\color{blue},
  numberstyle={\ttfamily},
%  breaklines=true,
  breakatwhitespace=true,
  tabsize=3,
  escapechar=|
}

\mdfsetup{font=\scriptsize}

\def\codesize{\smaller}
\def\<#1>{\codeid{#1}}
\newcommand{\codeid}[1]{\ifmmode{\mbox{\codesize\ttfamily{#1}}}\else{\codesize\ttfamily #1}\fi}

%****************enlarge layout
\textheight     243.5mm
\topmargin      -20.0mm
\textwidth      500pt
\hoffset        -80pt
%*****************theorems and such
\newcounter{esnu}
\newenvironment{esercizio}{\medskip \noindent {\bf Esercizio\addtocounter{esnu}{1} \arabic{esnu}}}{}
\pagestyle{empty}
\newcommand{\liff}{\mathrel{\leftrightarrow}}   % Logical IFF Symbol
\newcommand{\metaiff}{\Longleftrightarrow}      %iff in metatheory

\begin{document}

\begin{center}
  \textbf{Esame di Programmazione II, 20 febbraio 2025}\\
  (si consegni \<AbstractNote.java>, \<AbstractSong.java>, \<BasicSong.java>, \<EnglishNote.java>, \<IllegalNoteException.java>, \<ItalianNote.java> e \<Scale.java>)
\end{center}

\emph{
Si crei un progetto Eclipse e
il package \<it.univr.notes>. Si copino al suo interno
le classi del compito.
Non si modifichino le dichiarazioni dei metodi e delle classi. Si possono definire altri campi,
metodi o costruttori non richiesti dal compito, ma devono essere \<private>. Si possono definire altre classi, che in tal caso vanno consegnate.
La soluzione che verr\`a consegnata dovr\`a compilare,
altrimenti non verr\`a corretta.}

\vspace*{2ex}

L'interfaccia \<Note>, gi\`a completa,
rappresenta una nota musicale tramite il suo semitono: note basse
hanno semitono basso e note acute hanno semitono alto.
Le note italiane e inglesi si differenziano per come
il loro \<toString> le rappresenta, secondo la seguente tabella:
%
\begin{center}
  \begin{tabular}{|c|c|c|}
    \hline
    \<ItalianNote> & semitono & \<EnglishNote>\\\hline\hline
    1.do & 0 & 1.C \\\hline
    1.do\# & 1 & 1.C\# \\\hline
    1.re & 2 & 1.D \\\hline
    1.re\# & 3 & 1.D\# \\\hline
    1.mi & 4 & 1.E \\\hline
    1.fa & 5 & 1.F \\\hline
    1.fa\# & 6 & 1.F\# \\\hline
    1.sol & 7 & 1.G \\\hline
    1.sol\# & 8 & 1.G\# \\\hline
    1.la & 9 & 1.A \\\hline
    1.la\# & 10 & 1.A\# \\\hline
    1.si & 11 & 1.B \\\hline\hline
    2.do & 12 & 2.C \\\hline
    2.do\# & 13 & 2.C\# \\\hline
    2.re & 14 & 2.D \\\hline
    2.re\# & 15 & 2.D\# \\\hline
    2.mi & 16 & 2.E \\\hline
    2.fa & 17 & 2.F \\\hline
    2.fa\# & 18 & 2.F\# \\\hline
    2.sol & 19 & 2.G \\\hline
    2.sol\# & 20 & 2.G\# \\\hline
    2.la & 21 & 2.A \\\hline
    2.la\# & 22 & 2.A\# \\\hline
    2.si & 23 & 2.B \\\hline\hline
    3.do & 24 & 3.C \\\hline
    3.do\# & 25 & 3.C\# \\\hline
    eccetera & eccetera & eccetera
  \end{tabular}
\end{center}
%
Le note hanno un metodo \<shift> che permette di ottenere una nota
a partire da un'altra nota,
aggiungendo o togliendo semitoni. Per esempio, (1.la).\<shift>(3) restituisce 2.do,
mentre (1.G).\<shift>(8) restituisce 2.D\# e (2.do).\<shift>(-1) restituisce 1.si.

\newpage

%\vspace*{2ex}
\textbf{Esercizio 1 ($2$ punti).}
Si definisca l'eccezione controllata \<IllegalNoteException> che viene
usata per indicare che si sta cercando di creare una nota fuori dai limiti
$0\ldots\mathtt{Note.MAX\_SEMITONE}$.

\vspace*{1ex}\textbf{Esercizio 2 ($3$ punti).}
Si completi la classe \<AbstractNote.java>, che contiene il codice comune alle
implementazioni concrete delle note.

\vspace*{1ex}\textbf{Esercizio 3 ($5$ punti).}
Si completino le sottoclassi \<ItalianNote.java> e \<EnglishNote.java>, che si differenziano
per \<toString> e \<shift>. Quest'ultimo metodo se applicato a una
\<ItalianNote> restituisce una \<ItalianNote> e se applicato a una \<EnglishNote>
restituisce una \<EnglishNote>.

\vspace*{1ex}\textbf{Esercizio 4 ($5$ punti).}
L'interfaccia \<Song> (gi\`a completa, da non modificare) rappresenta una canzone,
cio\`e una sequenza di note. Iterando su un \<Song> si devono ottenere le sue note.
Si completi la classe \<AbstractSong.java> con il codice comune alle
implementazioni concrete delle canzoni, cio\`e il \<toString>, che concatena
il \<toString> delle note della canzone con uno spazio come separatore.

\vspace*{1ex}\textbf{Esercizio 5 ($7$ punti).}
Si completi la classe \<BasicSong> che implementa una canzone le cui note
sono fornite esplicitamente al costruttore e salvate dentro una lista di note.

\vspace*{1ex}\textbf{Esercizio 6 ($9$ punti).}
Si completi la classe \<Scale> che implementa una canzone le cui note
sono ottenute come una scala che parte da una nota \<start> fornita al costruttore
e sale di semitono in semitono, per un totale di $12$ note. Si chiede di risolvere questo punto
senza aggiungere campi alla classe \<Scale>, oltre a quelli che si trovano l\`i gi\`a definiti.

\vspace*{2ex}
\hrule
\vspace*{2ex}

Per capire meglio, si consideri il
seguente \<Main.java>, gi\`a scritto e da non modificare:
%
\begin{lstlisting}[language=Java]
public class Main {

  public static void main(String[] args) throws IllegalNoteException {
    // una canzone fatta da 4 note
    Song s1 = new BasicSong
      (new ItalianNote(5), new ItalianNote(25), new EnglishNote(13), new EnglishNote(38));
    // una scala dalla nota inglese di semitono 20 in su, fino alla nota inglese di semitono 31
    Song s2 = new Scale(new EnglishNote(20));
    System.out.println("s1 = " + s1); // stampa s1
    System.out.println("s2 = " + s2); // stampa s2
    System.out.println("s1.shift(-3) = " + s1.shift(-3)); // stampa s1 meno tre semitoni
    System.out.println("s2.shift(4) = " + s2.shift(4)); // stampa s2 piu' quattro semitoni
    System.out.println("s1 = " + s1); // ristampa s1 (non sara' cambiato)
    System.out.println("s2 = " + s2); // ristampa s2 (non sara' cambiato)
    s2.shift(29); // eccezione: spostando l'ultima nota di s2 si esce dai limiti 0...MAX_SEMITONE
  }
}
\end{lstlisting}
%
La sua esecuzione dovrebbe stampare:
%
\begin{mdframed}[backgroundcolor=lightred]
  {\small\begin{verbatim}
1 = 1.fa 3.do# 2.C# 4.D
s2 = 2.G# 2.A 2.A# 2.B 3.C 3.C# 3.D 3.D# 3.E 3.F 3.F# 3.G
s1.shift(-3) = 1.re 2.la# 1.A# 3.B
s2.shift(4) = 3.C 3.C# 3.D 3.D# 3.E 3.F 3.F# 3.G 3.G# 3.A 3.A# 3.B
s1 = 1.fa 3.do# 2.C# 4.D
s2 = 2.G# 2.A 2.A# 2.B 3.C 3.C# 3.D 3.D# 3.E 3.F 3.F# 3.G
Exception in thread "main" it.univr.notes.IllegalNoteException
\end{verbatim}}
\end{mdframed}

\end{document}
