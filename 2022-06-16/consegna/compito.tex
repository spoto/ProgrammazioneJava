\documentclass[12pt]{article}
\usepackage[pdftex]{graphicx}
\usepackage{amsfonts}
\usepackage[italian]{babel}
\usepackage{graphicx}
\usepackage{color}
\usepackage{multirow,bigdelim}
\usepackage{relsize}
\usepackage{fdsymbol}

\definecolor{grey}{rgb}{0.3,0.3,0.3}

\usepackage{listings, framed}
\lstset{
  language=Java,
  showstringspaces=false,
  columns=flexible,
  basicstyle={\small\ttfamily},
  frame=none,
  numbers=none,
  keywordstyle=\bfseries\color{grey},
  commentstyle=\itshape\color{red},
  identifierstyle=\color{black},
  stringstyle=\color{blue},
  numberstyle={\ttfamily},
%  breaklines=true,
  breakatwhitespace=true,
  tabsize=3,
  escapechar=|
}

\def\codesize{\smaller}
\def\<#1>{\codeid{#1}}
\newcommand{\codeid}[1]{\ifmmode{\mbox{\codesize\ttfamily{#1}}}\else{\codesize\ttfamily #1}\fi}

%****************enlarge layout
\textheight     243.5mm
\topmargin      -20.0mm
\textwidth      480pt
\hoffset        -80pt
%*****************theorems and such
\newcounter{esnu}
\newenvironment{esercizio}{\medskip \noindent {\bf Esercizio\addtocounter{esnu}{1} \arabic{esnu}}}{}
\pagestyle{empty}
\newcommand{\liff}{\mathrel{\leftrightarrow}}   % Logical IFF Symbol
\newcommand{\metaiff}{\Longleftrightarrow}      %iff in metatheory

\begin{document}

\begin{center} {\bf Esame di Programmazione II, 16 giugno 2022}\end{center}

\emph{
Si crei un progetto Eclipse e
il package \<it.univr.dadi>. Si copino al suo interno
le classi del compito.
Non si modifichino le dichiarazioni dei metodi. Si possono definire altri campi,
metodi, costruttori e classi, ma devono essere \<private>.
La soluzione che verr\`a consegnata dovr\`a compilare,
altrimenti non verr\`a corretta.}

\vspace*{2ex}

\begin{center}{\Large Esercizio 1} ($5$ punti)\\
  \textbf{(si consegni \<Dado.java>)}
\end{center}

Si completi la classe astratta \<Dado.java>. Essa
rappresenta un dado con un numero prefissato di facce, che pu\`o essere lanciato
ottenendo un numero tra uno e il suo numero di facce (inclusi).

\vspace*{2ex}

\begin{center}{\Large Esercizio 2} ($5$ punti)\\
  \textbf{(si consegnino \<D6.java>, \<D8.java> e \<D10.java>)}
\end{center}

Si creino tre sottoclassi concrete di \<Dado.java>, che rappresentano
rispettivamente un dado a sei facce (\<D6.java>), un dado a otto facce
(\<D8.java>) e un dado a dieci facce (\<D10.java>).

\vspace*{2ex}

\begin{center}{\Large Esercizio 3} ($15$ punti)\\
  \textbf{(si consegni \<Lanci.java>)}
\end{center}
  
Si completi la classe \<Lanci.java>,
che rappresenta l'esecuzione di pi\`u lanci con dei dadi.
Il numero dei lanci da effettuare e i dadi da usare sono forniti al costruttore.
Tale costruttore dovr\`a lanciare i dadi per il numero di lanci richiesto,
tenendo traccia dei numeri ottenuti (quando si lanciano pi\`u dadi, il numero
ottenuto \`e la somma dei numeri ottenuti da ciascun dado).
Il metodo \<toString()> restituisce i numeri ottenuti dal costruttore, tra parentesi quadre
e separati da virgole. Per esempio, lanciando dieci volte due dadi con sei facce
si potrebbe ottenere la stringa \<[4, 2, 6, 12, 2, 9, 8, 8, 7, 11]> chiamando
\<toString()>. Il metodo \<frequenze()> restituisce una stringa che descrive
i numeri ottenuti dal costruttore come istogrammi di asterischi, di lunghezza proporzionale alla
frequenza del numero ottenuto, seguiti dalla frequenza percentuale con cui il numero
\`e stato ottenuto. Si vedano gli esempi alla pagina seguente.

\vspace*{2ex}

\begin{center}{\Large Esercizio 4} ($6$ punti)\\
  \textbf{(si consegni \<LanciBarreDiverse.java>)}
\end{center}

Si completi la sottoclasse \<LanciBarreDiverse.java> di \<Lanci.java>, che si comporta
in modo identico a \<Lanci.java> ma stampa le barre degli istogrammi con tre caratteri
diversi, come nell'ultimo esempio della pagina seguente.

\vspace*{2ex}
\hrule
\vspace*{2ex}

Se tutto \`e corretto, l'esecuzione di \<Main.java> deve stampare qualcosa del tipo:

{\scriptsize\begin{verbatim}
Lanciamo 20 volte due dadi a sei facce
Lanci ottenuti: [6, 6, 8, 5, 8, 8, 8, 8, 9, 7, 6, 7, 9, 7, 4, 4, 8, 3, 2, 3]
  2: ***** (5.0%)
  3: ********** (10.0%)
  4: ********** (10.0%)
  5: ***** (5.0%)
  6: *************** (15.0%)
  7: *************** (15.0%)
  8: ****************************** (30.0%)
  9: ********** (10.0%)
 10:  (0.0%)
 11:  (0.0%)
 12:  (0.0%)

Lanciamo 10000 volte un dado a sei facce e uno a dieci facce
  2: * (1.5%)
  3: *** (3.4%)
  4: **** (5.0%)
  5: ******* (7.3%)
  6: ******** (8.2%)
  7: ********* (9.7%)
  8: ********** (10.3%)
  9: ********* (9.9%)
 10: ********* (9.6%)
 11: ********** (10.1%)
 12: ******** (8.5%)
 13: ****** (6.3%)
 14: ***** (5.2%)
 15: *** (3.4%)
 16: * (1.7%)

Lanciamo 10000 volte un dado a otto facce
  1: ************ (12.5%)
  2: ************ (12.4%)
  3: ************ (12.2%)
  4: ************ (12.3%)
  5: ************ (12.3%)
  6: ************* (13.4%)
  7: ************ (12.8%)
  8: ************ (12.1%)

Lanciamo 10000 volte tre dadi a sei facce
  3:  (0.5%)
  4: * (1.5%)
  5: ** (2.7%)
  6: **** (4.7%)
  7: ****** (7.0%)
  8: ********* (9.5%)
  9: *********** (11.5%)
 10: ************ (12.8%)
 11: ************ (12.6%)
 12: *********** (11.6%)
 13: ********* (9.4%)
 14: ****** (6.8%)
 15: **** (4.6%)
 16: ** (2.9%)
 17: * (1.4%)
 18:  (0.5%)

Lanciamo 10000 volte tre dadi a sei facce, usando barre diverse
  3:  (0.4%)
  4: @ (1.4%)
  5: ++ (3.0%)
  6: **** (4.6%)
  7: @@@@@@@ (7.3%)
  8: +++++++++ (9.6%)
  9: *********** (11.6%)
 10: @@@@@@@@@@@@ (12.5%)
 11: ++++++++++++ (12.1%)
 12: ************ (12.0%)
 13: @@@@@@@@@ (9.4%)
 14: +++++++ (7.1%)
 15: **** (4.5%)
 16: @@ (2.7%)
 17: + (1.4%)
 18:  (0.5%)

\end{verbatim}}

\end{document}
