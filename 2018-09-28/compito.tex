\documentclass[12pt]{article}
\usepackage[pdftex]{graphicx}
\usepackage{amsfonts}
\usepackage[italian]{babel}
\usepackage{graphicx}
\usepackage{color}

\definecolor{grey}{rgb}{0.3,0.3,0.3}

\usepackage{listings, framed}
\lstset{
  language=Java,
  showstringspaces=false,
  columns=flexible,
  basicstyle={\small\ttfamily},
  frame=none,
  numbers=none,
  keywordstyle=\bfseries\color{grey},
  commentstyle=\itshape\color{red},
  identifierstyle=\color{black},
  stringstyle=\color{blue},
  numberstyle={\ttfamily},
%  breaklines=true,
  breakatwhitespace=true,
  tabsize=3,
  escapechar=|
}

%****************enlarge layout
\textheight     243.5mm
\topmargin      -20.0mm
\textwidth      480pt
\hoffset        -80pt
%*****************theorems and such
\newcounter{esnu}
\newenvironment{esercizio}{\medskip \noindent {\bf Esercizio\addtocounter{esnu}{1} \arabic{esnu}}}{}
\pagestyle{empty}
\newcommand{\liff}{\mathrel{\leftrightarrow}}   % Logical IFF Symbol
\newcommand{\metaiff}{\Longleftrightarrow}      %iff in metatheory

\begin{document}

%\begin{tabular}{llclcr}
% \hspace{-35pt} &{\bf COGNOME:} & \hspace{100pt}        &{\bf NOME:}    & \hspace{100pt}        &{\bf MATRICOLA:}%\hspace{35pt} \\
%\hline
%\end{tabular}
\begin{center} {\bf Esame di Programmazione II, 28 settembre 2018}\end{center}
%\`

Le note musicali sono distinte in $12$ semitoni e vengono scritte diversamente
fra italiano e inglese, come si vede nella tabella seguente:
%
\begin{center}
  \begin{tabular}{|c|c|c|}
    \hline
    \textbf{semitono} & \textbf{italiano} & \textbf{inglese} \\\hline\hline
    0  & do    & C   \\\hline
    1  & do\#  & C\# \\\hline
    2  & re    & D   \\\hline
    3  & re\#  & D\# \\\hline
    4  & mi    & E   \\\hline
    5  & fa    & F   \\\hline
    6  & fa\#  & F\# \\\hline
    7  & sol   & G   \\\hline
    8  & sol\# & G\# \\\hline
    9  & la    & A   \\\hline
    10 & la\#  & A\# \\\hline
    11 & si    & B   \\\hline
  \end{tabular}
\end{center}
%
Il simbolo \# in italiano si pronuncia \emph{diesis} e in inglese \emph{sharp}.

\begin{esercizio}~\textbf{[6 punti]}
  Si completi la seguente classe astratta che implementa una nota,
  genericamente, senza ancora distinguere fra italiana e inglese:
%
\begin{lstlisting}
public abstract class Note implements Comparable<Note> {
  protected final int semitone;

  // inizializza la nota per il semitono indicato;
  // se il semitono non e' fra 0 e 11, deve lanciare una IllegalArgumentException
  protected Note(int semitone) { ... }

  @Override public abstract String toString();

  // due note sono uguali se e solo se hanno lo stesso semitono
  @Override public final boolean equals(Object other) { ... }

  // compatibile con equals() e non banale
  @Override public final int hashCode() { ... }

  // le note sono ordinate per semitono crescente
  @Override public final int compareTo(Note other) { ... }
}
\end{lstlisting}
\end{esercizio}

\begin{esercizio}~\textbf{[8 punti]}
  Si scriva la sottoclasse concreta \texttt{ItalianNote} di \texttt{Note},
  il cui metodo \texttt{toString()} usa la rappresentazione italiana delle note (\texttt{do}, \texttt{do\#}, \ldots).
  Tale sottoclasse dovr\`a definire $12$ costanti pubbliche \texttt{DO},
  \texttt{DO\_DIESIS}, \texttt{RE}, \ldots, \texttt{SI} di tipo \texttt{ItalianNote}, che contengono le note
  con tali nomi. Tali costanti devono essere gli unici oggetti
  di classe \texttt{ItalianNote} disponibili, nel senso che l'utilizzatore di \texttt{ItalianNote}
  non deve potere costruirne altri. Si definisca similmente la sottoclasse \texttt{EnglishNote} di
  \texttt{Note}, il cui metodo \texttt{toString()} usa la rappresentazione inglese delle note.
  Tale classe dovr\`a definire $12$ costanti pubbliche \texttt{C}, \texttt{C\_SHARP}, \texttt{D}, \ldots,
  \texttt{B} di tipo \texttt{EnglishNote} e dovr\`a impedire la costruzione di altre \texttt{EnglishNote}.
\end{esercizio}

\begin{esercizio}~\textbf{[5 punti]}
  Si completi la seguente classe di test JUnit:
\begin{lstlisting}
public class NoteTest {
  // asserisce che il toString() del do italiano e' la stringa "do"
  @Test public void testToString() { ... }

  // asserisce che l'italiano re# e l'inglese D# sono equals()
  @Test public void testEquals() { ... }

  // asserisce che l'italiano re# e l'inglese D# hanno lo stesso hashCode()
  @Test public void testHashCode() { ... }

  // asserisce che l'italiano re# e l'inglese E hanno hashCode() diversi
  @Test public void testHashCodeNonTrivial() { ... }

  // asserisce che l'italiano do viene prima dell'inglese D
  @Test public void testCompareTo() { ... }
}
\end{lstlisting}
\end{esercizio}

\begin{esercizio}~\textbf{[11 punti]}
  Si completi la seguente classe, che rappresenta una canzone: contiene il testo
  della canzone (di una sola riga) e permette di posizionare delle note sul testo:
\begin{lstlisting}
public class Song {
  ...
  public Song(String lyrics) { ... } // lyrics e' il testo della canzone (una riga)

  // posiziona la nota alla posizione indicata. Deve lanciare una IllegalArgumentException
  // se position non e' dentro il testo della canzone. Deve lanciare una
  // NoteAtSamePositionException se c'e' gia' una nota alla posizione indicata
  public void add(Note note, int position) { ... }

  @Override public String toString() { ... }
}
\end{lstlisting}
\end{esercizio}

\begin{esercizio}~\textbf{[2 punti]}
  Si definisca l'eccezione non controllata \texttt{NoteAtSamePositionException}.
\end{esercizio}
%
\mbox{}\\
\hrule
\mbox{}\\

Se tutto \`e  corretto, l'esecuzione del seguente codice:
%
\begin{lstlisting}
Song yellowSubmarine = new Song("In the town where I was born lived a man who sailed the sea");
yellowSubmarine.add(EnglishNote.C_SHARP, 7);  // C# su "town"
yellowSubmarine.add(EnglishNote.C_SHARP, 56); // C# su "sea"
yellowSubmarine.add(EnglishNote.F_SHARP, 24); // F# su "born"
yellowSubmarine.add(EnglishNote.G_SHARP, 37); // G# su "man"
// yellowSubmarine.add(EnglishNote.A, 24); // -> NoteAtSamePositionException
System.out.println(yellowSubmarine);
\end{lstlisting}
%
dovr\`a stampare:
%
\begin{verbatim}
       C#               F#           G#                 C# 
In the town where I was born lived a man who sailed the sea
\end{verbatim}

\begin{center}
\textbf{Si possono definire altri campi, metodi, costruttori e classi, ma solo \texttt{private}.}
\end{center}

\end{document}
