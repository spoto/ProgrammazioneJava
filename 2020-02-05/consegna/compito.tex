\documentclass[12pt]{article}
\usepackage[pdftex]{graphicx}
\usepackage{amsfonts}
\usepackage[italian]{babel}
\usepackage{graphicx}
\usepackage{color}
\usepackage{multirow,bigdelim}
\usepackage{relsize}
\usepackage{fdsymbol}

\definecolor{grey}{rgb}{0.3,0.3,0.3}

\usepackage{listings, framed}
\lstset{
  language=Java,
  showstringspaces=false,
  columns=flexible,
  basicstyle={\small\ttfamily},
  frame=none,
  numbers=none,
  keywordstyle=\bfseries\color{grey},
  commentstyle=\itshape\color{red},
  identifierstyle=\color{black},
  stringstyle=\color{blue},
  numberstyle={\ttfamily},
%  breaklines=true,
  breakatwhitespace=true,
  tabsize=3,
  escapechar=|
}

\def\codesize{\smaller}
\def\<#1>{\codeid{#1}}
\newcommand{\codeid}[1]{\ifmmode{\mbox{\codesize\ttfamily{#1}}}\else{\codesize\ttfamily #1}\fi}

%****************enlarge layout
\textheight     243.5mm
\topmargin      -20.0mm
\textwidth      480pt
\hoffset        -80pt
%*****************theorems and such
\newcounter{esnu}
\newenvironment{esercizio}{\medskip \noindent {\bf Esercizio\addtocounter{esnu}{1} \arabic{esnu}}}{}
\pagestyle{empty}
\newcommand{\liff}{\mathrel{\leftrightarrow}}   % Logical IFF Symbol
\newcommand{\metaiff}{\Longleftrightarrow}      %iff in metatheory

\begin{document}

%\begin{tabular}{llclcr}
% \hspace{-35pt} &{\bf COGNOME:} & \hspace{100pt}        &{\bf NOME:}    & \hspace{100pt}        &{\bf MATRICOLA:}%\hspace{35pt} \\
%\hline
%\end{tabular}
\begin{center} {\bf Esame di Programmazione II, 5 febbraio 2020}\end{center}
%\`

\emph{
Si crei un progetto Eclipse e, nella directory dei sorgenti,
si crei il package \texttt{it.univr.cards}. Si copi al suo interno
le classi del compito.
Non si modifichino le dichiarazioni dei metodi. Si possono definire altri campi,
metodi, costruttori e classi, ma devono essere \texttt{private}.
Questo significa che eventuali classi aggiuntive potranno solo essere interne o anonime.
La consegna fornita compila.
Anche la soluzione che verr\`a consegnata dovr\`a compilare,
altrimenti non verr\`a corretta.
Alla fine si consegnino soltanto \<Card.java>, \<Deck.java>, \<Ranking.java> e \<Main.java>,
separati (cio\`e non in un file zip).
}

\mbox{}\\

\begin{esercizio}~\textbf{[6 punti]}
  La classe \texttt{Card} rappresenta una carta del gioco del poker,
  fatta da un valore (enumerazione \<Value>, gi\`a fatta e da non modificare: 2,3,4,5,6,7,8,9,10,J,Q,K,1: si noti che l'asso ha il valore massimo)
  e da un seme (enumerazione \<Suit>, gi\`a fatta e da non modificare: $\spadesuit,\clubsuit,\vardiamondsuit,\varheartsuit$).
  La si completi dove indicato con \texttt{TODO}.
\end{esercizio}

\begin{esercizio}~\textbf{[3 punti]}
  L'enumerazione \<Ranking> rappresenta il ranking, o punteggio,
  di una mano di poker, cio\`e di cinque carte
  diverse del mazzo da poker. Si completi questa enumerazione in modo da enumerare le possibilit\`a,
  oltre a \<NONE>, nel seguente ordine crescente per ranking:
  \begin{description}
  \item[\<THREE\_OF\_KIND>:] in italiano sarebbe \emph{tris}, cio\`e tre carte di ugual valore e le altre di valore diverso, come per esempio $[10\vardiamondsuit, J\spadesuit, Q\clubsuit, J\varheartsuit, J\clubsuit]$
  \item[\<STRAIGHT>:] in italiano sarebbe \emph{scala}, cio\`e cinque carte non dello stesso seme che si possono mettere in sequenza, come per esempio $[10\vardiamondsuit, 1\spadesuit, K\spadesuit, Q\varheartsuit, J\varheartsuit]$
    (si ricordi che l'asso ha il valore massimo)
  \item[\<FLUSH>:] in italiano sarebbe \emph{colore}, cio\`e cinque carte dello stesso seme ma non scala, come per esempio $[5\spadesuit, 6\spadesuit, 10\spadesuit, J\spadesuit, K\spadesuit]$
  \item[\<FULL\_HOUSE>:] in italiano sarebbe \emph{full}, cio\`e un tris e una coppia, come per esempio $[1\spadesuit, 3\clubsuit, 3\varheartsuit, 1\clubsuit, 3\vardiamondsuit]$
  \item[\<FOUR\_OF\_KIND>:] in italiano sarebbe \emph{poker}, cio\`e quattro carte dello stesso valore, come per esempio $[K\vardiamondsuit, K\spadesuit, K\clubsuit, 7\vardiamondsuit, K\varheartsuit]$
  \item[\<STRAIGHT\_FLUSH>:] in italiano sarebbe \emph{scala reale}, cio\`e una scala con tutte le carte dello stesso seme, come per esempio $[9\varheartsuit, Q\varheartsuit, J\varheartsuit, 10\varheartsuit, K\varheartsuit]$
  \end{description}

\end{esercizio}

\begin{esercizio}~\textbf{[13 punti]}
  La classe \<Deck> rappresenta una mano di poker fatta da cinque carte diverse.
  La si completi dove indicato con \<TODO>. In particolare, vanno completati
  i costruttori e i metodi di test del ranking, come
  ad esempio \<isStraight()>, che determinano se le cinque carte hanno il ranking corrispondente.
  Per esempio, \<isStraight()> determina se le cinque carte formano una scala.
  Va completato anche il metodo \<getRanking()>, che restituisce il ranking della
  mano, e il metodo \<hasRankingFrom()>, che determina se la mano ha almeno un dato ranking minimo
  indicato.

  \mbox{}\\
  
  \noindent
  \textbf{Suggerimenti:}
  \begin{enumerate}
  \item non dimenticatevi di lanciare l'eccezione prevista nel secondo costruttore di \<Deck>
    (guardate i commenti prima di esso)
  \item la classe \<Deck> dovr\`a in qualche modo contenere le cinque carte.
    Riflettete su quale struttura dati sia la pi\`u semplice per i vostri scopi
    e su come pu\`o convenirvi mantenere le cinque carte al suo interno;
  \item leggete bene i commenti del codice riportati sopra i metodi di test, come
    \<isStraight()>. Tali commenti sono la descrizione di una soluzione semplice
    per implementare i metodi. Pu\`o aiutarvi implementare qualche metodo privato di supporto?
  \end{enumerate}
\end{esercizio}

\begin{esercizio}~\textbf{[10 punti]}
  Modificate la classe \texttt{Main} dove indicato con \<TODO>,
  in modo da farle stampare un output simile a quello che trovate in \<stampa.txt>:
  \begin{enumerate}
  \item prima genera e stampa $4$ full house a caso
  \item poi $6$ four of a kind a caso
  \item poi $8$ straight a caso
  \item poi $5$ straight flush a caso
  \item poi $9$ three of a kind a caso
  \item poi $5$ flush a caso
  \item poi $10$ mani di poker a caso con ranking da flush in su
  \item infine crea tre mani \<d1>, \<d2> e \<d3> e verifica che siano tutte e tre dei tris.
  \end{enumerate}
  Questa classe chiama ripetutamente il suo metodo \<print(howMany, when)>,
  che dovrete completare.
  Tale metodo stampa \<howMany> esempi di \<Deck> a caso che soddisfano la condizione
  \<when>. Tale condizione \`e un'istanza dell'interfaccia di libreria (gi\`a esistente)
  \<java.lang.function.Predicate$\langle$Deck$\rangle$>, che ha un unico metodo \<test(Deck)>
  per sapere se un deck soddisfa o meno il predicato.

  \mbox{}\\
  
  \noindent
  \textbf{Suggerimento:}
  Dovrete passare a \<print(howMany, when)> vostre implementazioni dell'interfaccia di libreria
  \<java.lang.function.Predicate$\langle$Deck$\rangle$>. Ci sono tanti modi per farlo.
  Ricordatevi che quell'interfaccia ha un unico metodo.
\end{esercizio}

\end{document}
