\documentclass[12pt]{article}
\usepackage{amssymb}
\usepackage{amsmath}
\usepackage{color}
\usepackage{graphicx}
\usepackage{mdframed}
\usepackage{listings, xcolor}
\usepackage{textcomp}

\definecolor{verylightgray}{rgb}{.97,.97,.97}

\definecolor{myBlue}{rgb}{0.5,0.5,1}
\definecolor{myLightBlue}{rgb}{0.35,0.6,0.8}
\definecolor{myBlack}{rgb}{0,0,0}
\definecolor{myGreen}{rgb}{0.1,0.6,0.2}
\definecolor{myGray}{rgb}{0.5,0.5,0.5}
\definecolor{myLightgray}{rgb}{0.95,0.95,0.95}
\definecolor{myMauve}{rgb}{0.58,0,0.82}
\definecolor{grey}{rgb}{0.3,0.3,0.3}
\definecolor{lightgrey}{rgb}{0.9,0.9,0.9}

\thispagestyle{empty}
\setlength{\textwidth}{18.5cm}
\setlength{\topmargin}{-2.5cm}
\setlength{\textheight}{24.5cm}
\setlength{\oddsidemargin}{-1cm}
\setlength{\evensidemargin}{-1cm}

\begin{document}
\begin{center}{\LARGE Esame di Programmazione 2}\\
\begin{center}
  \large 2 febbraio 2022, turno delle 11:30 (tempo disponibile: 2 ore)
\end{center}
\end{center}

Gli identificatori di un linguaggio di programmazione sono spesso scritti
in \emph{snake-style}, in cui le parole sono separate da
underscore, come ad esempio \texttt{snakes\_Are\_sloW\_food}
(il maiuscolo o minuscolo \`e a libera scelta del programmatore).
Si immagini un altro modo di scrivere gli identificatori, in cui le parole
vengono concatenate, le vocali si scrivono maiuscole e le lettere minuscole.
Per esempio \texttt{snAkEsArEslOwfOOd}. Questo modo alternativo di scrivere
gli identificatori lo chiameremo \emph{vowel-style}.
Entrambi sono esempio di identificatori \emph{multiword}, cio\`e composti
a partire da pi\`u parole.

L'interfaccia che definisce un \emph{identificatore} ha un unico metodo che
permette di tradurre un identificare in stringa:

{\small\begin{verbatim}
public interface Identifier {
  String toString();
}
\end{verbatim}}

\vspace*{1ex}
\begin{center}{\Large Esercizio 1} ($13$ punti)\\
  \textbf{(si consegni \texttt{MultiWordIdentifier.java})}
\end{center}

Si completi la classe astratta \texttt{MultiWordIdentifier} che rappresenta
un identificatore multiword.
\textbf{\`E possibile aggiungere campi o metodi ma solamente se privati}.
Si noti la presenza di due metodi \texttt{toString}:
il primo richiamer\`a il secondo su ciascuna parola componente dell'identificatore
e concatener\`a il risultato in un'unica stringa.

{\small\begin{verbatim}
@Override public final String toString() {
  // TODO: si richiami il metodo ausiliario toString(pos, word) sulle componenti
  // e si concateni il risultato in un'unica stringa
  return "";
}

// restituisce la stringa con cui si stampa la componente pos-esima dell'identificatore
protected abstract String toString(int pos, String word);
\end{verbatim}}

\vspace*{1ex}
\begin{center}{\Large Esercizio 2} ($9$ punti)\\
  \textbf{(si consegni \texttt{VowelStyleIdentifier.java})}
\end{center}

Si completi la classe concreta \texttt{VowelStyleIdentifier}, sottoclasse
di \texttt{MultiWordIdentifier}. Si noti che la classe ha tre costruttori:
il primo costruisce l'identificatore a partire dalle stringhe componenti,
il secondo a partire da un qualsiasi iterabile e il terzo concatenando le
componenti di altri identificatori:

{\small\begin{verbatim}
public VowelStyleIdentifier(String... words) {
  // TODO
}

public VowelStyleIdentifier(Iterable<String> words) {
  // TODO
}

public VowelStyleIdentifier(MultiWordIdentifier... ids) {
  // TODO
}
\end{verbatim}}

\vspace*{1ex}
\begin{center}{\Large Esercizio 3} ($9$ punti)\\
  \textbf{(si consegni \texttt{SnakeStyleIdentifier.java})}
\end{center}

Si completi la classe concreta \texttt{SnakeStyleIdentifier}, sottoclasse
di \texttt{MultiWordIdentifier}, analogamente a quanto fatto per
\texttt{VowelStyleIdentifier}.

\begin{center}
  * * *
\end{center}

Per esempio, un'esecuzione del file \texttt{Main.java} fornito \`e:

\begin{mdframed}[backgroundcolor=lightgrey] 
\begin{verbatim}
Inserisci una parola alla volta e termina con END
ciao
Amico
comE
va
END
id1 = cIAOAmIcOcOmEvA
id2 = ciao_Amico_comE_va
id3 = cIAOAmIcOcOmEvAcIAOAmIcOcOmEvA
id3 snake style = ciao_Amico_comE_va_ciao_Amico_comE_va
\end{verbatim}
\end{mdframed}

\end{document}
