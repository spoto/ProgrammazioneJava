\documentclass{article}[10pt]
\usepackage[pdftex]{graphicx}
\usepackage{amsfonts}
\usepackage[italian]{babel}
\usepackage{graphicx}
\usepackage{color}

\definecolor{grey}{rgb}{0.3,0.3,0.3}

\usepackage{listings, framed}
\lstset{
  language=Java,
  showstringspaces=false,
  columns=flexible,
  basicstyle={\small\ttfamily},
  frame=none,
  numbers=none,
  keywordstyle=\bfseries\color{grey},
  commentstyle=\itshape\color{red},
  identifierstyle=\color{black},
  stringstyle=\color{blue},
  numberstyle={\ttfamily},
%  breaklines=true,
  breakatwhitespace=true,
  tabsize=3,
  escapechar=|
}

%****************enlarge layout
\textheight     243.5mm
\topmargin      -20.0mm
\textwidth      480pt
\hoffset        -80pt
%*****************theorems and such
\newcounter{esnu}
\newenvironment{esercizio}{\medskip \noindent {\bf Esercizio\addtocounter{esnu}{1} \arabic{esnu}}}{}
\pagestyle{empty}
\newcommand{\liff}{\mathrel{\leftrightarrow}}   % Logical IFF Symbol
\newcommand{\metaiff}{\Longleftrightarrow}      %iff in metatheory

\begin{document}

%\begin{tabular}{llclcr}
% \hspace{-35pt} &{\bf COGNOME:} & \hspace{100pt}        &{\bf NOME:}    & \hspace{100pt}        &{\bf MATRICOLA:}%\hspace{35pt} \\
%\hline
%\end{tabular}
\begin{center} {\bf Esame di Programmazione II, 27 febbraio 2018}\end{center}
%\`

Si consideri un'implementazione di un sistema di gestione degli esami universitari.
Agli esami \`e possibile iscrivere studenti e poi verbalizzare l'esito. La verbalizzazione
\`e possibile solo per studenti che siano gi\`a iscritti e non \`e possibile
verbalizzare due volte l'esito a uno studente. Per esempio, il seguente codice:

\begin{lstlisting}[frame=l,numbers=left]
public class Main {
  public static void main(String[] args)
      throws VerbalizzazioneGiaEffettuataException, StudenteNonIscrittoException {

    Esame esame = new EsamePerEsito();
    Studente s1 = new Studente(151535, "Giorgio", "Levi");
    Studente s2 = new Studente(200345, "Fausto", "Spoto");
    Studente s3 = new Studente(324422, "Albert", "Einstein");
    Studente s4 = new Studente(145066, "Camilla", "De Sanctis");
    Studente s5 = new Studente(111044, "Eleonora", "Botticelli");
    esame.iscrivi(s1, s2, s4, s5);
    esame.verbalizza(s1, Esito.VENTI);
    esame.verbalizza(s4, Esito.TRENTAELODE);
    esame.verbalizza(s2, Esito.RITIRATO);
    esame.verbalizza(s5, Esito.VENTI);
    System.out.println(esame);  |\label{line:println}|
    esame.verbalizza(s3, Esito.TRENTA);  |\label{line:exception}|
  }
}
\end{lstlisting}
%
deve stampare alla linea~\ref{line:println}:
%
\begin{lstlisting}
200345                Spoto               Fausto: rit
111044           Botticelli             Eleonora: 20
151535                 Levi              Giorgio: 20
145066           De Sanctis              Camilla: 30L
\end{lstlisting}  
%
e terminare alla linea~\ref{line:exception} con una \texttt{StudenteNonIscrittoException: Lo studente Albert Einstein non e' iscritto all'esame}.

L'enumerazione degli esiti di un esame \`e gi\`a scritta:
%
\begin{lstlisting}
public enum Esito {
  RITIRATO("rit"), INSUFFICIENTE("ins"), DICIOTTO("18"), DICIANNOVE("19"), VENTI("20"), VENTUNO("21"),
  VENTIDUE("22"), VENTITRE("23"), VENTIQUATTRO("24"), VENTICINQUE("25"), VENTISEI("26"),
  VENTISETTE("27"), VENTOTTO("28"), VENTINOVE("29"), TRENTA("30"), TRENTAELODE("30L");

  private final String name;
  private Esito(String name) { this.name = name; }
  @Override public String toString() { return name; }
}
\end{lstlisting}
%
Si ricordi che i suoi elementi sono gi\`a \texttt{Comparable<Esito>} secondo l'ordine
con cui sono enumerati nel codice.

\begin{esercizio}~\textbf{[5 punti]}
Si completi la seguente classe, che implementa uno studente. Uno studente \`e
\texttt{equals()} solo a un altro studente con la stessa matricola:
%
\begin{lstlisting}
public class Studente {
  public final int matricola;
  public final String nome;
  public final String cognome;
 
  public Studente(int matricola, String nome, String cognome) { ... }
  @Override public boolean equals(Object other) { ... }
  @Override public int hashCode() { ... }  // non deve ritornare banalmente una costante
}
\end{lstlisting}
%
\end{esercizio}

\begin{esercizio}~\textbf{[2 punti]}
  Si scrivano le due
  eccezioni controllate \texttt{VerbalizzazioneGiaEffettuataException}
  e \texttt{StudenteNonIscrittoException} il cui costruttore deve ricevere come argomento uno
  \texttt{Studente} per potere costruire un messaggio di eccezione che includa il suo nome e
  cognome (si veda l'esempio dell'eccezione generata dal \texttt{Main}).
\end{esercizio}

\begin{esercizio}~\textbf{[12 punti]}
  Si completi la seguente classe, che implementa un esame a cui gli
  studenti possono iscriversi e per il quale \`e possibile verbalizzare
  un esito dopo l'iscrizione:
%
\begin{lstlisting}
public abstract class Esame implements Iterable<Esame.Verbalizzazione> {
  private final Set<Studente> iscritti = new HashSet<>();
  private final SortedSet<Verbalizzazione> verbalizzazioni = ...; // si completi questa linea!

  // iscrive all'esame tutti gli studenti indicati
  public final void iscrivi(Studente... studenti) { ... }

  // verbalizza l'esito per uno studente;
  // - lancia una StudenteNonIscrittoException se lo studente non era iscritto all'esame
  // - lancia una VerbalizzazioneGiaEffettuataException se lo studente aveva gia' verbalizzato
  public final void verbalizza(Studente studente, Esito esito)
    throws VerbalizzazioneGiaEffettuataException, StudenteNonIscrittoException { ... }

  // ritorna la concatenazione del toString() delle verbalizzazioni, separate da "\n"
  @Override public final String toString() { ... }

  // ritorna l'iteratore sulle verbalizzazioni effettuate, nell'ordine dato da getComparator()
  @Override public final Iterator<Verbalizzazione> iterator() { ... }

  // ritorna il comparatore da usare per creare le verbalizzazioni: attenzione, e' abstract!
  protected abstract Comparator<Verbalizzazione> getComparator();

  public final static class Verbalizzazione {    // classe interna
    private final Studente studente;   private final Esito esito;
    private Verbalizzazione(Studente studente, Esito esito) { ... }
    public Studente getStudente() { ... }
    public Esito getEsito() { ... }

    // una Verbalizzazione e' equals() a un'altra che abbia stesso esito e stesso studente
    @Override public boolean equals(Object other) { ... }
    @Override public int hashCode() { ... }    // non deve ritornare banalmente una costante
    @Override public String toString() {
      return String.format("%6d %20s %20s: %s", studente.matricola, studente.cognome, studente.nome, esito); 
    }
  }
}
\end{lstlisting}
%
  Si noti che \`e possibile iterare
  su un \texttt{Esame}, ottenendo una dopo l'altra le verbalizzazioni
  effettuate, nell'ordine specificato dal comparatore restituito da
  \texttt{getComparator()}. Un comparatore infatti \`e una interfaccia
  di libreria che ha un unico metodo che
  specifica chi viene prima fra due elementi:
%
\begin{lstlisting}
public interface Comparator<T> {
  // restituisce un numero negativo se o1 viene prima di o2;
  // un numero positivo se o2 viene prima di o1; 0 se o1 e o2 sono uguali
  int compare(T o1, T o2);
}
\end{lstlisting}

\textbf{Suggerimento: } normalmente la classe di libreria \texttt{TreeSet<T>} realizza
un insieme ordinato di \texttt{T} e tale tipo generico deve implementare \texttt{Comparable<T>}.
\`E per\`o possibile definire dei \texttt{TreeSet<T>} anche per tipi \texttt{T} che non
implementino \texttt{Comparable<T>}, purch\'e il criterio di confronto venga fornito al
momento della costruzione dell'insieme, tramite un \texttt{Comparator<T>}, cos\`{\i}: \texttt{new TreeSet<T>(comparator)}.
\end{esercizio}

\begin{esercizio}~\textbf{[9 punti]}
%
  Si definisca una classe concreta \texttt{EsamePerMatricola} che estende
  \texttt{Esame} fissando come ordinamento delle verbalizzazioni
  quello crescente per matricola.
  Si definisca una classe concreta \texttt{EsamePerEsito} che estende
  \texttt{Esame} fissando come ordinamento delle verbalizzazioni
  quello crescente per esito e, a parit\`a di esito, quello crescente per matricola.
%
\end{esercizio}

\begin{esercizio}~\textbf{[3 punti]}
%
  Nella definizione della classe interna \texttt{Verbalizzazione} dell'Esercizio~3:
  \begin{enumerate}
  \item \`E possibile dichiarare tale classe \texttt{private} al posto che \texttt{public}? Perch\'e ?
  \item \`E possibile eliminare la parola chiave \texttt{static}? Cosa cambierebbe?
  \end{enumerate}
%
\end{esercizio}

%\begin{center}
%\textbf{\`E possibile definire campi, metodi, costruttori e classi aggiuntive, ma solo \texttt{private}.}
%\end{center}

\end{document}
