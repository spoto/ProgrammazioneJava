\documentclass[12pt]{article}
\usepackage[pdftex]{graphicx}
\usepackage{amsfonts}
\usepackage[italian]{babel}
\usepackage{graphicx}
\usepackage{color}
\usepackage{multirow,bigdelim}
\usepackage{relsize}
\usepackage{fdsymbol}
\usepackage{mdframed}

\definecolor{grey}{rgb}{0.3,0.3,0.3}
\definecolor{verylightgray}{rgb}{.97,.97,.97}
\definecolor{lightred}{rgb}{1,.70,.70}

\usepackage{listings, framed}
\lstset{
  language=Java,
  showstringspaces=false,
  columns=flexible,
  basicstyle={\small\ttfamily},
  frame=none,
  numbers=none,
  keywordstyle=\bfseries\color{grey},
  commentstyle=\itshape\color{red},
  identifierstyle=\color{black},
  stringstyle=\color{blue},
  numberstyle={\ttfamily},
%  breaklines=true,
  breakatwhitespace=true,
  tabsize=3,
  escapechar=|
}

\mdfsetup{font=\scriptsize}

\def\codesize{\smaller}
\def\<#1>{\codeid{#1}}
\newcommand{\codeid}[1]{\ifmmode{\mbox{\codesize\ttfamily{#1}}}\else{\codesize\ttfamily #1}\fi}

%****************enlarge layout
\textheight     243.5mm
\topmargin      -20.0mm
\textwidth      500pt
\hoffset        -80pt
%*****************theorems and such
\newcounter{esnu}
\newenvironment{esercizio}{\medskip \noindent {\bf Esercizio\addtocounter{esnu}{1} \arabic{esnu}}}{}
\pagestyle{empty}
\newcommand{\liff}{\mathrel{\leftrightarrow}}   % Logical IFF Symbol
\newcommand{\metaiff}{\Longleftrightarrow}      %iff in metatheory

\begin{document}

\begin{center}
  \textbf{Esame di Programmazione II, 9 luglio 2025}\\
  (si consegni \<NoSuchRowException.java>, \<AbstractFigure.java>, \<Triangle.java>, \<Circle.java> e \<Frame.java>)
\end{center}

\emph{
Si crei un progetto Eclipse e
il package \<it.univr.figures>. Si copino al suo interno
le classi del compito.
Non si modifichino le dichiarazioni dei metodi e delle classi. Si possono definire altri campi,
metodi o costruttori non richiesti dal compito, ma devono essere \<private>. Si possono definire altre classi,
che in tal caso vanno consegnate.
La soluzione che verr\`a consegnata dovr\`a compilare,
altrimenti non verr\`a corretta.}

\vspace*{2ex}

L'interfaccia \<Figure> rappresenta un figura che si pu\`o stampare. Essa ha un'altezza
(numero di caratteri verticalmente) e una larghezza (numero di caratteri orizzontalmente)
e un metodo \<getRow(int n)> che restituisce la riga $n$-esima della figura. Per esempio,
la seguente figura:

\begin{verbatim}
    *    questa riga e' getRow(0)
   * *   questa riga e' getRow(1)
  *   *  questa riga e' getRow(2)
 *     * questa riga e' getRow(3)
*********questa riga e' getRow(4)
\end{verbatim}

\noindent
rappresenta un triangolo di altezza 5 e larghezza 9. Le sue cinque righe sono restituite
dal metodo \<getRow>, come indicato sopra. Per esempio, \<getRow(2)> restituisce la stringa
\<"$\sqcup\sqcup$*$\sqcup\sqcup\sqcup$*$\sqcup\sqcup$">
(inclusi gli spazi all'inizio e alla fine, indicati qui con $\sqcup$,
ma senza nessuna andata a capo alla fine).
Ne consegue che il \<toString()> di una figura \`e la concatenzazione dei suoi \<getRow()>,
con una andata a capo aggiunta alla fine di ciascuno. Se \<getRow> viene chiamato fuori dai limiti
della figura, esso lancia una \<NoSuchRowException>. Per esempio, questo accadrebbe se nella
figura mostrata sopra si chiamasse \<getRow(5)> o \<getRow(6)> o \<getRow(-1)>.

Per capire meglio le classi che andranno implementate, si consideri il
seguente \<main>, gi\`a scritto e da non modificare:
%
\begin{lstlisting}[language=Java]
    for (int h = 1; h <= 5; h++) {
      System.out.println("triangolo di altezza " + h + ":");
      System.out.println(new Triangle(h));
    }
    for (int r = 1; r <= 16; r += 5) {
      System.out.println("cerchio di raggio " + r + ":");
      System.out.println(new Circle(r));
    }
    for (int h = 1; h <= 5; h++) {
      System.out.println("triangolo di altezza " + h + ", in cornice:");
      System.out.println(new Frame(new Triangle(h)));
    }
    for (int r = 1; r <= 16; r += 5) {
      System.out.println("cerchio di raggio " + r + ", in cornice doppia:");
      System.out.println(new Frame(new Frame(new Circle(r))));
    }
    // ok, nessuna eccezione
    new Frame(new Circle(2)).getRow(5);
    // eccezione, poiche' un cerchio di raggio 2 ha le righe da 0 a 3,
    // e aggiungendo una cornice le righe andranno da 0 a 5; quindi 6 e' fuori
    new Frame(new Circle(2)).getRow(6);
\end{lstlisting}
%
che dovrebbe stampare:
%
\begin{mdframed}[backgroundcolor=lightred]
  {\small\begin{verbatim}
triangolo di altezza 1:
*

triangolo di altezza 2:
 * 
***

triangolo di altezza 3:
  *  
 * * 
*****

triangolo di altezza 4:
   *   
  * *  
 *   * 
*******

triangolo di altezza 5:
    *    
   * *   
  *   *  
 *     * 
*********

cerchio di raggio 1:
**
**

cerchio di raggio 6:
    ****    
  **    **  
 *        * 
 *        * 
*          *
*          *
*          *
*          *
 *        * 
 *        * 
  **    **  
    ****    

cerchio di raggio 11:
        ******        
     ***      ***     
    **          **    
   *              *   
  *                *  
 **                ** 
 *                  * 
 *                  * 
*                    *
*                    *
*                    *
*                    *
*                    *
*                    *
 *                  * 
 *                  * 
 **                ** 
  *                *  
   *              *   
    **          **    
     ***      ***     
        ******        

cerchio di raggio 16:
            ********            
         ***        ***         
       **              **       
      **                **      
     *                    *     
    *                      *    
   *                        *   
  **                        **  
  *                          *  
 *                            * 
 *                            * 
 *                            * 
*                              *
*                              *
*                              *
*                              *
*                              *
*                              *
*                              *
*                              *
 *                            * 
 *                            * 
 *                            * 
  *                          *  
  **                        **  
   *                        *   
    *                      *    
     *                    *     
      **                **      
       **              **       
         ***        ***         
            ********            

triangolo di altezza 1, in cornice:
@@@
@*@
@@@

triangolo di altezza 2, in cornice:
@@@@@
@ * @
@***@
@@@@@

triangolo di altezza 3, in cornice:
@@@@@@@
@  *  @
@ * * @
@*****@
@@@@@@@

triangolo di altezza 4, in cornice:
@@@@@@@@@
@   *   @
@  * *  @
@ *   * @
@*******@
@@@@@@@@@

triangolo di altezza 5, in cornice:
@@@@@@@@@@@
@    *    @
@   * *   @
@  *   *  @
@ *     * @
@*********@
@@@@@@@@@@@

cerchio di raggio 1, in cornice doppia:
@@@@@@
@@@@@@
@@**@@
@@**@@
@@@@@@
@@@@@@

cerchio di raggio 6, in cornice doppia:
@@@@@@@@@@@@@@@@
@@@@@@@@@@@@@@@@
@@    ****    @@
@@  **    **  @@
@@ *        * @@
@@ *        * @@
@@*          *@@
@@*          *@@
@@*          *@@
@@*          *@@
@@ *        * @@
@@ *        * @@
@@  **    **  @@
@@    ****    @@
@@@@@@@@@@@@@@@@
@@@@@@@@@@@@@@@@

cerchio di raggio 11, in cornice doppia:
@@@@@@@@@@@@@@@@@@@@@@@@@@
@@@@@@@@@@@@@@@@@@@@@@@@@@
@@        ******        @@
@@     ***      ***     @@
@@    **          **    @@
@@   *              *   @@
@@  *                *  @@
@@ **                ** @@
@@ *                  * @@
@@ *                  * @@
@@*                    *@@
@@*                    *@@
@@*                    *@@
@@*                    *@@
@@*                    *@@
@@*                    *@@
@@ *                  * @@
@@ *                  * @@
@@ **                ** @@
@@  *                *  @@
@@   *              *   @@
@@    **          **    @@
@@     ***      ***     @@
@@        ******        @@
@@@@@@@@@@@@@@@@@@@@@@@@@@
@@@@@@@@@@@@@@@@@@@@@@@@@@

cerchio di raggio 16, in cornice doppia:
@@@@@@@@@@@@@@@@@@@@@@@@@@@@@@@@@@@@
@@@@@@@@@@@@@@@@@@@@@@@@@@@@@@@@@@@@
@@            ********            @@
@@         ***        ***         @@
@@       **              **       @@
@@      **                **      @@
@@     *                    *     @@
@@    *                      *    @@
@@   *                        *   @@
@@  **                        **  @@
@@  *                          *  @@
@@ *                            * @@
@@ *                            * @@
@@ *                            * @@
@@*                              *@@
@@*                              *@@
@@*                              *@@
@@*                              *@@
@@*                              *@@
@@*                              *@@
@@*                              *@@
@@*                              *@@
@@ *                            * @@
@@ *                            * @@
@@ *                            * @@
@@  *                          *  @@
@@  **                        **  @@
@@   *                        *   @@
@@    *                      *    @@
@@     *                    *     @@
@@      **                **      @@
@@       **              **       @@
@@         ***        ***         @@
@@            ********            @@
@@@@@@@@@@@@@@@@@@@@@@@@@@@@@@@@@@@@
@@@@@@@@@@@@@@@@@@@@@@@@@@@@@@@@@@@@

Exception in thread "main" it.univr.figures.NoSuchRowException
\end{verbatim}}
\end{mdframed}

\vspace*{2ex}\textbf{Esercizio 1 ($2$ punti).}
Si scriva la classe di eccezione \textbf{controllata} \<NoSuchRowException.java>.

\vspace*{1.5ex}\textbf{Esercizio 2 ($7$ punti).}
Si completi la classe astratta \<AbstractFigure.java>, con l'unico metodo \<toString()>.
Questo \<toString()> verr\`a ereditato da tutte le implementazioni concrete delle
figure, senza modifiche, quindi lo definiamo \<final>.

\vspace*{1.5ex}\textbf{Esercizio 3 ($7$ punti).}
Si completi la classe concreta \<Triangle.java> che implementa un triangolo rettangolo con il
vertice posizionato in alto e di altezza \<height>. Guardate le figure di esempio
stampate sopra per capire qual \`e la sua larghezza e come deve essere implementato
il suo \<getRow()>. Per quest'ultimo metodo, il suggerimento \`e di
distinguere i quattro casi: prima riga, ultima riga, righe in mezzo, altre righe.
Si suggerisce inoltre di usare il metodo delle stringhe \<s.repeat(n)>, che
restituisce la stringa \<s> ripetuta $n$ volte.

\vspace*{1.5ex}\textbf{Esercizio 4 ($7$ punti).}
Si completi la classe concreta \<Circle.java> che implementa un cerchio di raggio \<radius>. 
Il suo metodo \<getRow()> \`e gi\`a parzialmente scritto. Il suggerimento \`e di completarlo
seguendo i commenti.

\vspace*{1.5ex}\textbf{Esercizio 5 ($8$ punti).}
Si completi la classe concreta \<Frame.java> che implementa una figura ottenuta
decorando un'altra figura \<figure> con una cornice aggiuntiva di caratteri \<@> al suo intorno.
Per il \<getRow()>, il suggerimento \`e di distinguere il caso della prima e ultima riga
della figura risultante (con cornice) dal caso delle altre sue righe.

\end{document}
