\documentclass[12pt]{article}
\usepackage[pdftex]{graphicx}
\usepackage{amsfonts}
\usepackage[italian]{babel}
\usepackage{graphicx}
\usepackage{color}
\usepackage{multirow,bigdelim}
\usepackage{relsize}
\usepackage{fdsymbol}
\usepackage{mdframed}

\definecolor{grey}{rgb}{0.3,0.3,0.3}
\definecolor{verylightgray}{rgb}{.97,.97,.97}
\definecolor{lightred}{rgb}{1,.70,.70}

\usepackage{listings, framed}
\lstset{
  language=Java,
  showstringspaces=false,
  columns=flexible,
  basicstyle={\small\ttfamily},
  frame=none,
  numbers=none,
  keywordstyle=\bfseries\color{grey},
  commentstyle=\itshape\color{red},
  identifierstyle=\color{black},
  stringstyle=\color{blue},
  numberstyle={\ttfamily},
%  breaklines=true,
  breakatwhitespace=true,
  tabsize=3,
  escapechar=|
}

\mdfsetup{font=\scriptsize}

\def\codesize{\smaller}
\def\<#1>{\codeid{#1}}
\newcommand{\codeid}[1]{\ifmmode{\mbox{\codesize\ttfamily{#1}}}\else{\codesize\ttfamily #1}\fi}

%****************enlarge layout
\textheight     243.5mm
\topmargin      -20.0mm
\textwidth      500pt
\hoffset        -80pt
%*****************theorems and such
\newcounter{esnu}
\newenvironment{esercizio}{\medskip \noindent {\bf Esercizio\addtocounter{esnu}{1} \arabic{esnu}}}{}
\pagestyle{empty}
\newcommand{\liff}{\mathrel{\leftrightarrow}}   % Logical IFF Symbol
\newcommand{\metaiff}{\Longleftrightarrow}      %iff in metatheory

\begin{document}

\begin{center}
  \textbf{Esame di Programmazione II, 5 luglio 2023 (si consegni \<Main.java> e \<PunishableSet.java>)}
\end{center}

\emph{
Si crei un progetto Eclipse e
il package \<it.univr.sets>. Si copino al suo interno
le classi del compito.
Non si modifichino le dichiarazioni dei metodi e delle classi. Si possono definire altri campi,
metodi, costruttori e classi, ma devono essere \<private>.
C'\`e solo un metodo \<public> da aggiungere.
La soluzione che verr\`a consegnata dovr\`a compilare,
altrimenti non verr\`a corretta.}

\vspace*{2ex}

Un \emph{insieme punibile} \`e un insieme di elementi, in cui
ogni elemento ha associato un punteggio. Per default, il punteggio
inizialmente \`e $1000$. Per esempio il seguente codice:
%
\begin{lstlisting}[language=Java]
  String[] names = { "Fausto", "Samantha", "Giulio", "Giovanna" };
  PunishableSet<String> set = new PunishableSet<>(names); // tutti partono da 1000 punti
  System.out.println(set);
\end{lstlisting}
%
stampa:
%
\vspace{-0.5ex}
\begin{mdframed}[backgroundcolor=lightred]
{\small\begin{verbatim}
Giovanna: 1000 points
Fausto: 1000 points
Samantha: 1000 points
Giulio: 1000 points
\end{verbatim}}
\end{mdframed}
%
Esiste un altro costruttore che permette di specificare il punteggio iniziale
per gli elementi appena aggiunti a un insieme, per esempio $500$ per i nomi
che terminano in \<o> e $800$ per gli altri
(l'interfaccia di libreria \<ToIntFunction>$<$\<E>$>$ \`e una funzione da \<E> ad \<int>):
%
\begin{lstlisting}[language=Java]
  ToIntFunction<String> init = new ToIntFunction<>() {
    public int applyAsInt(String s) { return s.endsWith("o") ? 500 : 800; }
  };
  set = new PunishableSet<>(names, init); // partono da 500 o 800 punti
  set.add("Rocco"); // Rocco parte da 500 punti
  System.out.println(set);
\end{lstlisting}
%
stampa:
%
\vspace{-0.5ex}
\begin{mdframed}[backgroundcolor=lightred]
{\small\begin{verbatim}
Giovanna: 800 points
Fausto: 500 points
Rocco: 500 points
Samantha: 800 points
Giulio: 500 points
\end{verbatim}}
\end{mdframed}

L'aggiunta di elementi a un insieme si fa con il metodo \<add>, la rimozione con \<remove>.
La riduzione dei punti di un elemento si fa con \<punish> e l'aumento
con \<pardon>. Se i punti di un elemento raggiungono zero o diventano negativi, quell'elemento
viene eliminato. Ad esempio:
%
\begin{lstlisting}[language=Java]
  set = new PunishableSet<>(names); // tutti partono da 1000 punti
  set.punish("Fausto", 400); // Fausto perde 400 punti
  set.punish("Giulio", 1010); // Giulio ne perde 1010 e viene eliminato
  System.out.println(set);
\end{lstlisting}
%
stampa:
%
\vspace{-0.5ex}
\begin{mdframed}[backgroundcolor=lightred]
{\small\begin{verbatim}
Giovanna: 1000 points
Fausto: 600 points
Samantha: 1000 points
\end{verbatim}}
\end{mdframed}

\textbf{Esercizio 1 ($7$ punti).}
Si completino i tre costruttori di \<PunishableSet>, cio\`e i due che abbiamo
usato negli esempi precedenti e un altro che permette di specificare anche
del codice che deve essere eseguito ogni volta che viene aggiunto o rimosso
un elemento dall'insieme.

\vspace*{2ex}\textbf{Esercizio 2 ($14$ punti).}
Si completino i metodi \<add>, \<remove>, \<punish> e \<pardon> di \<PunishableSet>.
Si legga bene la loro specifica nel commento che precede la dichiarazione dei metodi.

\vspace*{2ex}\textbf{Esercizio 3 ($3$ punti).}
Si renda \<PunishableSet> iterabile, in modo che iterando su un tale oggetto
si ottengano gli elementi dentro l'insieme (si tratta di aggiungere un solo
metodo \<public>).

\vspace*{2ex}\textbf{Esercizio 4 ($7$ punti).}
Si completi il \<Main>, creando i \<PunishableSet> come richiesto nel codice.
Dovrebbe stampare qualcosa del tipo:
%
\begin{mdframed}[backgroundcolor=lightred]
{\small\begin{verbatim}
set1:
Aurora: 1000 points
Fausto: 500 points
Samantha: 1000 points

set2:
Aurora: 6 points
Samantha: 8 points

set3:
Aurora: 4 points
Samantha: 3 points

set4:
Giovanna: 1000 points
Aurora: 2000 points
Fausto: 1500 points
Samantha: 2000 points

adding Fausto
adding Samantha
adding Giulio
adding Giovanna
adding Aurora
removing Giovanna
removing Giulio
set5:
Aurora: 600 points
Fausto: 100 points
Samantha: 600 points

Exception in thread "main" java.lang.IllegalArgumentException
\end{verbatim}}
\end{mdframed}

\end{document}
