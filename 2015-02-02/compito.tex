\documentclass{article}[10pt]
\usepackage[pdftex]{graphicx}
\usepackage{amsfonts}
\usepackage[italian]{babel}
%****************enlarge layout
\textheight     243.5mm
\topmargin      -20.0mm
\textwidth      480pt
\hoffset        -80pt
%*****************theorems and such
\newcounter{esnu}
\newenvironment{esercizio}{\medskip \noindent {\bf Esercizio\addtocounter{esnu}{1} \arabic{esnu}}}{}
\pagestyle{empty}
\newcommand{\liff}{\mathrel{\leftrightarrow}}   % Logical IFF Symbol
\newcommand{\metaiff}{\Longleftrightarrow}      %iff in metatheory

\begin{document}

%\begin{tabular}{llclcr}
% \hspace{-35pt} &{\bf COGNOME:} & \hspace{100pt}        &{\bf NOME:}    & \hspace{100pt}        &{\bf MATRICOLA:}%\hspace{35pt} \\
%\hline
%\end{tabular}
\begin{center} {\bf Esame di Programmazione II, 2 febbraio 2015}\end{center}
%\`

La ditta Paranoid produce l'omonimo sistema operativo
per una linea di smartphone su cui \`e possibile installare applicazioni (\emph{app}).
Esistono degli \emph{store} che accumulano app e le rendono disponibili ai clienti. Nel corso degli
anni, il sistema operativo Paranoid ha subito delle evoluzioni, per cui esistono gi\`a quattro
versioni successive, in ordine cronologico:
%
{\small
\begin{verbatim}
public enum OS {
  LITTLE_LION,   // prima versione
  RUNNING_TIGER,
  ACKWARD_BEAR,
  FLOATING_HIPPO // ultima versione
}
\end{verbatim}
}

\noindent
In Java, gli \texttt{enum} implementano \texttt{java.lang.Comparable}, per cui ad esempio gli elementi
di \texttt{OS} si possono comparare con il metodo \texttt{int compareTo(OS other)}.

\begin{esercizio}
\textbf{[5 punti]}
%
Si completi la seguente classe, che implementa un'app per il sistema Paranoid:
%
{\small
\begin{verbatim}
public class App implements Comparable<App> {
  ... aggiungete campi private
  public final OS MIN_OS;

  // minOS e' la versione minima di sistema operativo richiesta da questa app
  public App(String name, OS minOS, int version) {
    this.MIN_OS = minOS;
    ...
  }

  public String getName() { ...restituisce il nome dell'app }
  public int getVersion() { ...restituisce la versione dell'app }
  public String toString() { ...restituisce il nome dell'app seguito dalla sua versione }
  public boolean equals(Object other) { ...due app sono uguali se hanno uguale nome e versione }
  public int hashCode() { ...deve essere non banale }
  public int compareTo(App other) { ...due app vengono prima ordinate per nome, poi per versione }
}
\end{verbatim}}

\end{esercizio}

\begin{esercizio}
\textbf{[7 punti]}
%
Un dispositivo (smartphone) permette di installare un numero arbitrario di app, purch\'e esse siano compatibili
con la sua versione installata del sistema operativo Paranoid. \`E possibile reinstallare un'app gi\`a
installata, ma solo se si passa a una versione successiva dell'app. In tal caso, la nuova
versione sostituisce la vecchia versione della app. \`E sempre possibile deinstallare un'app.

Si completi la seguente classe, che implementa un dispositivo:
%
{\small
\begin{verbatim}
public class Device {
  public final OS os;
  ... eventuali campi privati

  // il dispositivo implementa il sistema operativo indicato e all'inizio non ha app installate
  public Device(OS os) {
    this.os = os;
  }

  // aggiunge l'app a quelle gia' installate, se possibile:
  // * se il sistema operativo del dispositivo non supporta l'app, lancia una AppNotSupportedException
  // * se l'app e' gia' installata in una versione non precedente, lancia una AppAlreadyInstalledException
  // * se l'app e' gia' installata in una versione precedente, la sostituisce con la nuova versione
  public void install(App app) { ... }

  // rimuove l'app da quelle gia' installate. Se non era installata, non fa nulla
  public void deInstall(App app) { ... }

  // restituisce le app installate su questo dispositivo
  public SortedSet<App> getInstalledApps() { ... }
}
\end{verbatim}}

\noindent
\textbf{Suggerimento:} pu\`o tornare utile la classe \texttt{java.util.TreeSet<E>}, che implementa
un insieme ordinato di \texttt{E} (che deve essere un \texttt{Comparable}) e che implementa l'interfaccia \texttt{java.util.SortedSet<E>}.
\end{esercizio}

\begin{esercizio}
\textbf{[2 punti]}
Sia \texttt{AppNotSupportedException} che \texttt{AppAlreadyInstalledException} sono sottoclassi
concrete dell'eccezione astratta \texttt{InstallationException}, sottoclasse di
\texttt{java.lang.RuntimeException}. Si scrivano queste tre classi di eccezione.
\end{esercizio}

\begin{esercizio}
\textbf{[8 punti]}
Uno store contiene app e ha l'intelligenza per selezionare quelle compatibili con un dato dispositivo.
Si completi la seguente classe che implementa uno store:
%
{\small
\begin{verbatim}
public class Store {
  private final String name;
  ... eventuali campi private

  public Store(String name) {
    this.name = name;
  }

  public void add(App app) {  ... aggiunge un'app a questo store }

  public String toString() {
    ... restituisce il nome dello store seguito dal toString() delle app contenute in questo store
  }

  public SortedSet<App> getAppsFor(Device device) {
    ... restituisce le app di questo store compatibili con il sistema operativo del dispositivo indicato. 
    Se piu' versioni sono compatibili col dispositivo, restituisce solo la versione piu' recente
  }
}
\end{verbatim}}

\end{esercizio}

\end{document}
