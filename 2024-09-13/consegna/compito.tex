\documentclass[12pt]{article}
\usepackage[pdftex]{graphicx}
\usepackage{amsfonts}
\usepackage[italian]{babel}
\usepackage{graphicx}
\usepackage{color}
\usepackage{multirow,bigdelim}
\usepackage{relsize}
\usepackage{fdsymbol}
\usepackage{mdframed}

\definecolor{grey}{rgb}{0.3,0.3,0.3}
\definecolor{verylightgray}{rgb}{.97,.97,.97}
\definecolor{lightred}{rgb}{1,.70,.70}

\usepackage{listings, framed}
\lstset{
  language=Java,
  showstringspaces=false,
  columns=flexible,
  basicstyle={\small\ttfamily},
  frame=none,
  numbers=none,
  keywordstyle=\bfseries\color{grey},
  commentstyle=\itshape\color{red},
  identifierstyle=\color{black},
  stringstyle=\color{blue},
  numberstyle={\ttfamily},
%  breaklines=true,
  breakatwhitespace=true,
  tabsize=3,
  escapechar=|
}

\mdfsetup{font=\scriptsize}

\def\codesize{\smaller}
\def\<#1>{\codeid{#1}}
\newcommand{\codeid}[1]{\ifmmode{\mbox{\codesize\ttfamily{#1}}}\else{\codesize\ttfamily #1}\fi}

%****************enlarge layout
\textheight     243.5mm
\topmargin      -20.0mm
\textwidth      500pt
\hoffset        -80pt
%*****************theorems and such
\newcounter{esnu}
\newenvironment{esercizio}{\medskip \noindent {\bf Esercizio\addtocounter{esnu}{1} \arabic{esnu}}}{}
\pagestyle{empty}
\newcommand{\liff}{\mathrel{\leftrightarrow}}   % Logical IFF Symbol
\newcommand{\metaiff}{\Longleftrightarrow}      %iff in metatheory

\begin{document}

\begin{center}
  \textbf{Esame di Programmazione II, 13 settembre 2024}\\
  (si consegni \<Corso.java>, \<Esame.java>, \<MainEsame.java>, \<Studente.java>, \<StudenteIllegaleException.java> e \<StudenteLavoratore.java>)
\end{center}

\emph{
Si crei un progetto Eclipse e
il package \<it.univr.corso>. Si copino al suo interno
le classi del compito.
Non si modifichino le dichiarazioni dei metodi e delle classi. Si possono definire altri campi,
metodi o costruttori non richiesti dal compito, ma devono essere \<private>. Si possono definire altre classi,
che in tal caso vanno consegnate.
La soluzione che verr\`a consegnata dovr\`a compilare,
altrimenti non verr\`a corretta.}

\vspace*{2ex}

\vspace*{2ex}\textbf{Esercizio 1 ($3$ punti).}
Si completi la classe \<Corso.java>, che rappresenta un corso di laurea (per esempio, \emph{Informatica}),
con un nome e una durata in anni.

\vspace*{2ex}\textbf{Esercizio 2 ($2$ punti).}
Si crei la classe di eccezione controllata \<StudenteIllegaleException.java>, che ha solo un costruttore, il quale riceve il messaggio di eccezione e lo passa alla superclasse.

\vspace*{2ex}\textbf{Esercizio 3 ($8$ punti).}
Si completi la classe \<Studente.java> che rappresenta uno studente, con un nome, cognome,
matricola e anno di iscrizione.
Non si dimentichi di implementare il lancio dell'eccezione dentro il costruttore.

\vspace*{2ex}\textbf{Esercizio 4 ($4$ punti).}
Si completi la classe \<StudenteLavoratore.java>, sottoclasse di \<Studente.java>. Rispetto a uno
sudente \emph{normale}, uno studente
lavoratore impiega il doppio degli anni per finire fuori corso (per esempio, sei anni per un corso
che abbia una durata di tre anni). Serve forse ridefinire qualche metodo della superclasse?

\vspace*{2ex}\textbf{Esercizio 5 ($7$ punti).}
Si completi la classe \<Esame.java>, che rappresenta un esame di un corso di laurea,
a cui si possono iscrivere degli studenti (per esempio, \emph{Programmazione Quantistica}).
Non si dimentichi di implementare il lancio dell'eccezione dentro il metodo \<iscrivi>.

\vspace*{2ex}\textbf{Esercizio 6 ($7$ punti).}
Si completi la classe di prova \<MainEsame.java>, nei tre punti indicati. Nel primo
va letto uno studente da tastiera, nel secondo e terzo vanno selezionati
solo alcuni degli studenti iscritti all'esame e poi ne fanno stampate le matricole (secondo punto)
o l'intero studente (terzo punto).

\vspace*{2ex}\textbf{Suggerimento 1:} per conoscere l'anno in cui ci troviamo, si pu\`o
usare \<Year.now().getValue()>, dove \<Year> \`e la classe di libreria \<java.time.Year>.

\vspace*{2ex}\textbf{Suggerimento 2:} in questo compito non serve usare n\'e lambda espressioni n\'e stream,
ma \`e ovviamente possibile farlo se ritenete che cos\`{\i} si semplifichi il codice. In tal caso,
\`e possibile concatenare le stringhe di uno stream di stringhe \<s>, andando a capo
tra una e l'altra stringa, scrivendo
\<s.collect(Collectors.joining("\textbackslash n"))>, dove \<Collectors> \`e la classe di libreria
\<java.util.stream.Collectors>.

\noindent
\rule{\textwidth}{1pt}

\begin{center}
  \textbf{ESEMPIO DI ESECUZIONE ALLA PAGINA SEGUENTE $\Rightarrow$}
\end{center}

\newpage

Se tutto \`e corretto, una esecuzione di \<MainEsame.java> potrebbe essere:

\begin{mdframed}[backgroundcolor=lightred]
  {\small\begin{verbatim}
Nome: Alessandro
Cognome: Manzoni
Matricola: 123456
Anno di immatricolazione: 2025

Studente illegale, riprova

Nome: Alessandro
Cognome: Manzoni
Matricola: 123456
Anno di immatricolazione: 2024

Esame di Programmazione Quantistica del corso di Informatica:
34555 Giulio Andreotti immatricolato nel 2017
98034 Giordano Bruni immatricolato nel 2018
111564 Giulio Rossi immatricolato nel 2021
123456 Alessandro Manzoni immatricolato nel 2024
151535 Antonietta Reale immatricolato nel 2020
178066 Alessandra Allegri immatricolato nel 2024

Matricole degli studenti fuori corso:
34555
151535

Studenti lavoratori:
34555 Giulio Andreotti immatricolato nel 2017
98034 Giordano Bruni immatricolato nel 2018
\end{verbatim}}
\end{mdframed}
%
Si noti che l'inserimento di
Alessandro Manzoni avviene da tastiera e che la prima volta tale inserimento
viene rifiutato perch\'e non \`e possibile
creare studenti immatricolari nel 2025, visto che al momento siamo nel 2024.

\end{document}
