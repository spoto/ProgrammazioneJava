\documentclass{article}[10pt]
\usepackage[pdftex]{graphicx}
\usepackage{amsfonts}
\usepackage[italian]{babel}
\usepackage{graphicx}

%****************enlarge layout
\textheight     243.5mm
\topmargin      -20.0mm
\textwidth      480pt
\hoffset        -80pt
%*****************theorems and such
\newcounter{esnu}
\newenvironment{esercizio}{\medskip \noindent {\bf Esercizio\addtocounter{esnu}{1} \arabic{esnu}}}{}
\pagestyle{empty}
\newcommand{\liff}{\mathrel{\leftrightarrow}}   % Logical IFF Symbol
\newcommand{\metaiff}{\Longleftrightarrow}      %iff in metatheory

\begin{document}

%\begin{tabular}{llclcr}
% \hspace{-35pt} &{\bf COGNOME:} & \hspace{100pt}        &{\bf NOME:}    & \hspace{100pt}        &{\bf MATRICOLA:}%\hspace{35pt} \\
%\hline
%\end{tabular}
\begin{center} {\bf Esame di Programmazione II, 6 settembre 2016}\end{center}
%\`

\begin{esercizio}
\textbf{[8 punti]}
Un \texttt{PhoneBook} \`e una raccolta di record contenenti nome, cognome,
numero di telefono e genere (maschile/femminile) di persone. Si completi
la sua implementazione, riempiendo le parti mancanti nella classe che segue:

{\scriptsize\begin{verbatim}
public class PhoneBook extends View {
  ...
  public static class Entry {
    public final String name;      public final String surname;
    public final int phone;        public final boolean sex;
    public final static boolean MALE = false, FEMALE = true;

    private Entry(String name, String surname, int phone, boolean sex) {
      this.name = name;    this.surname = surname;
      this.phone = phone;  this.sex = sex;
    }

    @Override public String toString() {
      return name + " " + surname + ": " + phone + (sex == MALE ? " [male]" : " [female]");
    }
  }

  public void add(String name, String surname, int phone, boolean sex) { ...
    // aggiunge l'entry; se gia' ne esisteva una con lo stesso nome e cognome, la sostituisce
  }

  public void remove(String name, String surname) { ...
    // rimuove l'entry con tale nome e cognome; se non e' presente, lancia una UnknownEntryException
  }

  @Override public Iterator<Entry> iterator() { ...
    // resituisce un iteratore sulle entry di questo PhoneBook
  }
}
\end{verbatim}}

\end{esercizio}

\begin{esercizio}
\textbf{[2 punti]}
Si implementi l'eccezione \texttt{UnknownEntryException}.
\end{esercizio}

\begin{esercizio}
\textbf{[10 punti]}
Una \texttt{View} \`e un iterabile di entry, cio\`e un oggetto che, se iterato, fornisce
un'entry alla volta. La sua implementazione \`e la seguente. Si noti che si tratta
di una classe astratta quindi il suo metodo \texttt{iterator()} deve essere
implementato nelle sue sottoclassi concrete, come ad esempio il \texttt{PhoneBook}
visto sopra:

{\scriptsize\begin{verbatim}
public abstract class View implements Iterable<Entry> {
  protected View() {}

  @Override public final String toString() {
    String result = "";
    for (Entry entry: this)
      result += entry.toString() + "\n";

    return result;
  }
}
\end{verbatim}}

\noindent
Si definisca una sottoclasse concreta \texttt{SexView} di \texttt{View} che, data
un'altra \texttt{View} padre, \`e un iterabile sulle sole entry del padre che abbiano
un genere ben preciso, fornito al costruttore. Si completi a tal fine la seguente
implementazione:

{\scriptsize\begin{verbatim}
public class SexView extends View {
  private final View parent;  private final boolean sex;
  public SexView(View parent, boolean sex) { this.parent = parent;  this.sex = sex; }
  @Override public Iterator<Entry> iterator() { ... }
}
\end{verbatim}}

\end{esercizio}

\begin{esercizio}
\textbf{[11 punti, solo per chi non ha gi\`a fatto il progetto degli anni passati]}
Si definisca una sottoclasse concreta \texttt{SortedView} di \texttt{View} che, data
un'altra \texttt{View} padre, \`e un iterabile sulle stesse entry del padre, ma nell'ordine
specificato da un comparatore fornito al costruttore. Si completi a tal fine la seguente
implementazione:

{\scriptsize\begin{verbatim}
public class SortedView extends View {
  private final View parent;  private final Comparator<Entry> comparator;

  public SortedView(View parent, Comparator<Entry> comparator) {
    this.parent = parent; this.comparator = comparator;
  }

  @Override public Iterator<Entry> iterator() { ... }
}
\end{verbatim}}

\noindent
Si ricordi che l'interfaccia di libreria \texttt{java.util.Comparator} \`e la seguente:
{\scriptsize\begin{verbatim}
public interface Comparator<T> {
  int compare(T o1, T o2);
}
\end{verbatim}}

\noindent
dove il metodo \texttt{compare()} ritorna un numero negativo se \texttt{o1} viene
prima di \texttt{o2}, ritorna un numero positivo se \texttt{o2} viene prima di \texttt{o1}
e ritorna $0$ altrimenti.

Si ricordi inoltre che una \texttt{java.util.List entries} pu\`o essere convertita in array
con l'istruzione:

{\scriptsize\begin{verbatim}
  Entry[] array = entries.toArray(new Entry[entries.size()])
\end{verbatim}}

\noindent
e che un \texttt{array} pu\`o essere ordinato rispetto a un comparatore con l'istruzione:
{\scriptsize\begin{verbatim}
  java.util.Arrays.sort(array, comparator)
\end{verbatim}}
\end{esercizio}

\hrule

\mbox{}\\

Se tutto \`e corretto, l'esecuzione del programma:
{\scriptsize\begin{verbatim}
public class Main {
  public static void main(String[] args) {
    PhoneBook book = new PhoneBook();
    book.add("Primo", "Levi", 7569222, MALE);
    book.add("Fausto", "Spoto", 1234567, MALE);
    book.add("Giorgio", "Levi", 1563423, MALE);
    book.add("Catherine", "Bach", 367745, FEMALE);
    book.add("Pietro", "Ferrara", 7865634, MALE);
    book.add("Alberto", "Lovato", 8746728, MALE);
    book.add("Thomas", "Scudiero", 453678, MALE);
    book.add("Pietro", "Ferrara", 333334, MALE); // sostituisce
    book.add("Cybill", "Shepherd", 987567546, FEMALE);
    book.add("Audrey", "Hepburn", 32444, FEMALE);
    book.remove("Fausto", "Spoto");

    View view = book;
    System.out.println("DIRECT VIEW");
    System.out.println(view);

    view = new SexView(view, MALE);
    System.out.println("MALE ONLY VIEW");
    System.out.println(view);

    Comparator<Entry> comparator = new Comparator<Entry>() { // ordina per cognome e poi per nome

      @Override public int compare(Entry address1, Entry address2) {
        int diff = address1.surname.compareTo(address2.surname);
        if (diff != 0)
          return diff;
        else
          return address1.name.compareTo(address2.name);
      }
    };

    view = new SortedView(view, comparator);
    System.out.println("SORTED MALE ONLY VIEW");
    System.out.println(view);
  }
}
\end{verbatim}}

\noindent dovr\`a stampare:

{\scriptsize\begin{verbatim}
DIRECT VIEW
Primo Levi: 7569222 [male]
Giorgio Levi: 1563423 [male]
Catherine Bach: 367745 [female]
Alberto Lovato: 8746728 [male]
Thomas Scudiero: 453678 [male]
Pietro Ferrara: 333334 [male]
Cybill Shepherd: 987567546 [female]
Audrey Hepburn: 32444 [female]

MALE ONLY VIEW
Primo Levi: 7569222 [male]
Giorgio Levi: 1563423 [male]
Alberto Lovato: 8746728 [male]
Thomas Scudiero: 453678 [male]
Pietro Ferrara: 333334 [male]

SORTED MALE ONLY VIEW
Pietro Ferrara: 333334 [male]
Giorgio Levi: 1563423 [male]
Primo Levi: 7569222 [male]
Alberto Lovato: 8746728 [male]
Thomas Scudiero: 453678 [male]            
\end{verbatim}}

\begin{center}
\textbf{\`E possibile definire campi, metodi, costruttori e classi aggiuntive, ma solo \texttt{private}.}
\end{center}

\end{document}
