\documentclass[12pt]{article}
\usepackage[pdftex]{graphicx}
\usepackage{amsfonts}
\usepackage[italian]{babel}
\usepackage{graphicx}
\usepackage{color}
\usepackage{multirow,bigdelim}

\definecolor{grey}{rgb}{0.3,0.3,0.3}

\usepackage{listings, framed}
\lstset{
  language=Java,
  showstringspaces=false,
  columns=flexible,
  basicstyle={\small\ttfamily},
  frame=none,
  numbers=none,
  keywordstyle=\bfseries\color{grey},
  commentstyle=\itshape\color{red},
  identifierstyle=\color{black},
  stringstyle=\color{blue},
  numberstyle={\ttfamily},
%  breaklines=true,
  breakatwhitespace=true,
  tabsize=3,
  escapechar=|
}

%****************enlarge layout
\textheight     243.5mm
\topmargin      -20.0mm
\textwidth      480pt
\hoffset        -80pt
%*****************theorems and such
\newcounter{esnu}
\newenvironment{esercizio}{\medskip \noindent {\bf Esercizio\addtocounter{esnu}{1} \arabic{esnu}}}{}
\pagestyle{empty}
\newcommand{\liff}{\mathrel{\leftrightarrow}}   % Logical IFF Symbol
\newcommand{\metaiff}{\Longleftrightarrow}      %iff in metatheory

\begin{document}

%\begin{tabular}{llclcr}
% \hspace{-35pt} &{\bf COGNOME:} & \hspace{100pt}        &{\bf NOME:}    & \hspace{100pt}        &{\bf MATRICOLA:}%\hspace{35pt} \\
%\hline
%\end{tabular}
\begin{center} {\bf Esame di Programmazione II, 29 gennaio 2019}\end{center}
%\`

\emph{
Si crei un progetto Eclipse e, nella directory dei sorgenti,
si crei il package \texttt{it.univr.rent}. Si copi al suo interno
le classi del compito, tranne \texttt{Main.java} che va copiato
dentro il package di default.
Se si realizzano nuove classi, le si creino dentro
il package \texttt{it.univr.rent}.
Non si modifichino le dichiarazioni dei metodi. Si possono definire altri campi,
metodi, costruttori e classi, ma devono essere \texttt{private}.
La consegna fornita compila (tranne il \texttt{Main} per il quale mancano delle classi).
Anche la soluzione che verr\`a consegnata dovr\`a compilare (incluso il \texttt{Main}),
altrimenti non verr\`a corretta.
}

\mbox{}\\

L'interfaccia \texttt{Model} (gi\`a completa, non la si modifichi)
rappresenta un modello di un veicolo che pu\`o essere preso a noleggio.
Ha metodi per conoscerne il nome, il prezzo giornaliero di noleggio e per sapere
se pu\`o essere guidato con una data patente. I tipi delle patenti sono definiti
dall'enumerazione \texttt{License} (gi\`a completa, non la si modifichi). Si noti
che i quattro tipi di patente permettono di guidare tipi diversi di veicoli.

\begin{esercizio}~\textbf{[9 punti]}
  La classe astratta \texttt{AbstractModel} implementa \texttt{Model} e
  fornisce il codice di quasi tutti i suoi metodi, nonch\'e un costruttore.
  La si modifichi dove indicato con \texttt{TODO}.
\end{esercizio}

\begin{esercizio}~\textbf{[6 punti]}
  Si definiscano quattro sottoclassi concrete \texttt{Motorbike}, \texttt{Car}, \texttt{Truck} e
  \texttt{Bus} di \texttt{AbstractModel}, che implementano i metodi lasciati abstract in
  \texttt{AbstractModel} e che
  forniscono un costruttore che richiede il nome del modello e il suo prezzo giornaliero di noleggio.
  Per esempio, si definisca \texttt{public Car(String name, int pricePerDay)} dentro \texttt{Car}.
\end{esercizio}

\begin{esercizio}~\textbf{[2 punti]}
  Si implementino due classi di eccezione \texttt{IllegalLicenseException} e
  \texttt{ModelNotAvailableException}, sottoclassi di \texttt{java.lang.IllegalArgumentException}.
\end{esercizio}

\begin{esercizio}~\textbf{[14 punti]}
  Si completi la classe \texttt{Agency}, che implementa un'agenzia di noleggio
  di veicoli.
  La si modifichi dove indicato con \texttt{TODO}.
\end{esercizio}

\mbox{}\\

\emph{Se tutto \`e  corretto, l'esecuzione del \texttt{Main}
  (gi\`a completo, non lo si modifichi)
  stamper\`a quanto riportato nel file di testo \texttt{risultato\_main.txt}.
}

\end{document}
