\documentclass{article}[10pt]
\usepackage[pdftex]{graphicx}
\usepackage{amsfonts}
\usepackage[italian]{babel}
%****************enlarge layout
\textheight     243.5mm
\topmargin      -20.0mm
\textwidth      480pt
\hoffset        -80pt
%*****************theorems and such
\newcounter{esnu}
\newenvironment{esercizio}{\medskip \noindent {\bf Esercizio\addtocounter{esnu}{1} \arabic{esnu}}}{}
\pagestyle{empty}
\newcommand{\liff}{\mathrel{\leftrightarrow}}   % Logical IFF Symbol
\newcommand{\metaiff}{\Longleftrightarrow}      %iff in metatheory

\begin{document}

%\begin{tabular}{llclcr}
% \hspace{-35pt} &{\bf COGNOME:} & \hspace{100pt}        &{\bf NOME:}    & \hspace{100pt}        &{\bf MATRICOLA:}%\hspace{35pt} \\
%\hline
%\end{tabular}
\begin{center} {\bf Esame di Programmazione II, 7 febbraio 2014}\end{center}
%\`

Si osservi il seguente programma, che utilizza delle classi che implementano un carrello della
spesa (classe \texttt{Cart})
nel quale \`e possibile inserire dei prodotti. Questi sono rappresentati dalla classe
\texttt{Product}. Esiste una sottoclasse \texttt{DiscountedProduct} che rappresenta un
prodotto scontato di una certa percentuale. Inoltre esiste la sottoclasse
\texttt{Buy2Take3Product} che rappresenta un prodotto offerto con la formula
del 3 per 2: paghi 2 e prendi 3. In altri termini, se ne ottiene uno gratis ogni tre acquistati.
Si noti che l'offerta del 3 per 2 non \`e cumulabile con ulteriori offerte.

{\small
\begin{verbatim}
public class Main {
  public static void main(String[] args) {
    Cart cart = new Cart();

    Product tv = new Product("Super Baffo", "TV al plasma", 589.99);
    Product cd = new Product("Roberto Malandrino", "CD di Laura Masini", 0.89);
    Product modem = new Product("Giorgio Rubacchioni", "modem ADSL", 30.40);

    cart.addProduct(tv);
    cart.addProducts(modem, cd);
    cart.addProduct(new DiscountedProduct(tv, 13.5));
    cart.addProduct(new DiscountedProduct(new DiscountedProduct(modem, 11.6), 15.9));
    cart.addProduct(tv, 2);

    // un cd offerto come prendi due e paghi tre!
    Product specialOffer = new Buy2Take3Product(cd);
    // ne compriamo sette esemplari
    cart.addProduct(specialOffer, 7);

    System.out.println(cart);

    // non posso scontare un 3 per 2: otterro' un'eccezione a questo punto
    cart.addProduct(new DiscountedProduct(specialOffer, 21.00));
  }
}
\end{verbatim}}

\noindent
Il programma dovrebbe stampare

{\small
\begin{verbatim}
TV al plasma    589.99 euro. Venduto da Super Baffo
modem ADSL    30.40 euro. Venduto da Giorgio Rubacchioni
CD di Laura Masini    0.89 euro. Venduto da Roberto Malandrino
TV al plasma [scontato del 13.50%]    510.34 euro. Venduto da Super Baffo
modem ADSL [scontato del 11.60%] [scontato del 15.90%]    22.60 euro. Venduto da Giorgio Rubacchioni
TV al plasma    589.99 euro. Venduto da Super Baffo
TV al plasma    589.99 euro. Venduto da Super Baffo
CD di Laura Masini    0.64 euro. Venduto da Roberto Malandrino
CD di Laura Masini    0.64 euro. Venduto da Roberto Malandrino
CD di Laura Masini    0.64 euro. Venduto da Roberto Malandrino
CD di Laura Masini    0.64 euro. Venduto da Roberto Malandrino
CD di Laura Masini    0.64 euro. Venduto da Roberto Malandrino
CD di Laura Masini    0.64 euro. Venduto da Roberto Malandrino
CD di Laura Masini    0.64 euro. Venduto da Roberto Malandrino

Exception in thread "main" java.lang.IllegalArgumentException: cannot reduce the price of CD di Laura Masini
\end{verbatim}}

\begin{esercizio}
\textbf{[5 punti]}
%
Si completi la seguente classe, che implementa un prodotto generico:
{\small
\begin{verbatim}
public class Product {
  ...
  public Product(String seller, String name, double price) { ... }
  protected Product(Product original) { ... crea una copia di original }
  public final String getSeller() { ...restituisce seller }
  public final String getName() { ...restituisce name }
  public double getPrice(Cart cart) { ...restituisce price }
  public boolean canBeReduced() { ... tutti i prodotti possono essere scontati meno il 3 per 2}
  @Override public String toString() {
    return getName();
  }
}
\end{verbatim}}

\noindent
Si noti che il metodo che restituisce il prezzo del prodotto richiede il carrello poich\'e, nelle sottoclassi, tale
prezzo potrebbe dipendere dagli altri prodotti presenti nel carrello (si pensi al 3 per 2).
\end{esercizio}

\begin{esercizio}
\textbf{[6 punti]}
%
Si completi la seguente classe, che rappresenta un prodotto scontato del \texttt{discount} per cento:
%
{\small
\begin{verbatim}
public class DiscountedProduct extends Product {
  ...
  public DiscountedProduct(Product original, double discount) {
    ... deve lanciare una IllegalArgumentException se original non puo' essere scontato
  }

  @Override public double getPrice(Cart cart) {
    ... si tenga conto dello sconto da applicare al prezzo del prodotto original
  }
  @Override public String toString() {
    ... si aggiunga "[scontato del XX%]" in fondo alla stampa di original
  }
}
\end{verbatim}}
\end{esercizio}

\begin{esercizio}
\textbf{[6 punti]}
Si completi la seguente classe, che rappresenta un prodotto offerto con il 3 per 2:
%
{\small
\begin{verbatim}
public class Buy2Take3Product extends Product {
  ...
  public Buy2Take3Product(Product original) {
    ... deve lanciare una IllegalArgumentException se original non puo' essere scontato
  }

  @Override public boolean canBeReduced() { ... non puo' essere scontato! }
  @Override public double getPrice(Cart cart) {
    ... tenete conto di quanti altri prodotti uguali (==) ci sono nel carrello
  }
}
\end{verbatim}}

\end{esercizio}

\begin{esercizio}
\textbf{[5 punti]}
Si completi la seguente classe, che rappresenta il carrello della spesa, nel quale
\`e possibile aggiungere dei prodotti. Si noti che questa classe implementa
\texttt{java.lang.Iterable}, per cui \`e possibile iterare su un carrello, ottenendo i prodotti
contenuti, uno alla volta.
%
{\small
\begin{verbatim}
import java.util.ArrayList;
import java.util.Iterator;
import java.util.List;

public class Cart implements Iterable<Product> {
  // i prodotti contenuti in questo carrello
  private final List<Product> products = new ArrayList<Product>();

  public void addProduct(Product product) { ... aggiunge il prodotto indicato }
  public void addProducts(Product... products) { ... aggiunge tutti i prodotti indicati }
  public void addProduct(Product product, int howManyTimes) { ... aggiunge il prodotto howManyTimes volte }

  @Override public String toString() {
    String result = "";
    for (Product product: this)
      result += product + "    " + String.format("%.2f euro", product.getPrice(this))
         + ". Venduto da " + product.getSeller() + "\n";
    return result;
  }

  @Override public Iterator<Product> iterator() { ... }
}
\end{verbatim}}
\end{esercizio}

\end{document}
