\documentclass[12pt]{article}
\usepackage[pdftex]{graphicx}
\usepackage{amsfonts}
\usepackage[italian]{babel}
\usepackage{graphicx}
\usepackage{color}
\usepackage{multirow,bigdelim}

\definecolor{grey}{rgb}{0.3,0.3,0.3}

\usepackage{listings, framed}
\lstset{
  language=Java,
  showstringspaces=false,
  columns=flexible,
  basicstyle={\small\ttfamily},
  frame=none,
  numbers=none,
  keywordstyle=\bfseries\color{grey},
  commentstyle=\itshape\color{red},
  identifierstyle=\color{black},
  stringstyle=\color{blue},
  numberstyle={\ttfamily},
%  breaklines=true,
  breakatwhitespace=true,
  tabsize=3,
  escapechar=|
}

%****************enlarge layout
\textheight     243.5mm
\topmargin      -20.0mm
\textwidth      480pt
\hoffset        -80pt
%*****************theorems and such
\newcounter{esnu}
\newenvironment{esercizio}{\medskip \noindent {\bf Esercizio\addtocounter{esnu}{1} \arabic{esnu}}}{}
\pagestyle{empty}
\newcommand{\liff}{\mathrel{\leftrightarrow}}   % Logical IFF Symbol
\newcommand{\metaiff}{\Longleftrightarrow}      %iff in metatheory

\begin{document}

%\begin{tabular}{llclcr}
% \hspace{-35pt} &{\bf COGNOME:} & \hspace{100pt}        &{\bf NOME:}    & \hspace{100pt}        &{\bf MATRICOLA:}%\hspace{35pt} \\
%\hline
%\end{tabular}
\begin{center} {\bf Esame di Programmazione II, 4 febbraio 2019, turno 1}\end{center}
%\`

\emph{
Si crei un progetto Eclipse e, nella directory dei sorgenti,
si crei il package \texttt{it.univr.agenda}. Si copi al suo interno
le classi del compito, tranne \texttt{Main.java} che va copiato
dentro il package di default.
Se si realizzano nuove classi, le si creino dentro
il package \texttt{it.univr.agenda}.
Non si modifichino le dichiarazioni dei metodi. Si possono definire altri campi,
metodi, costruttori e classi, ma devono essere \texttt{private}.
La consegna fornita compila.
Anche la soluzione che verr\`a consegnata dovr\`a compilare,
altrimenti non verr\`a corretta.
}

\mbox{}\\

Un istante temporale della giornata viene rappresentato in Italia
con ore, minuti e secondi, separati dal carattere due punti, con le ore che
vanno da 0 a 23. Negli Stati Uniti, invece, si usa una
rappresentazione su dodici ore, da 1 a 12, e si distingue tra prima e seconda
parte della giornata tramite i suffissi AM e PM. Per esempio, alcuni istanti
sono rappresentati in questo modo in Italia e Stati Uniti:

\begin{center}
  \begin{tabular}{|c|c|}
    \hline
    Italia & Stati Uniti \\\hline\hline
    00:00:00 & 12:00:00AM \\\hline
    00:23:08 & 12:23:08AM \\\hline
    01:23:08 & 01:23:08AM \\\hline
    02:23:08 & 02:23:08AM \\\hline
    11:23:08 & 11:23:08AM \\\hline\hline
    12:00:00 & 12:00:00PM \\\hline
    12:23:08 & 12:23:08PM \\\hline
    13:23:08 & 01:23:08PM \\\hline
    14:23:08 & 02:23:08PM \\\hline
    23:23:08 & 11:23:08PM \\\hline
  \end{tabular}
\end{center}
%
Si faccia attenzione in particolare alla rappresentazione della mezzanotte
e del mezzogiorno usata negli Stati Uniti, che pu\`o essere
sorprendente e indurre in errore.

\begin{esercizio}~\textbf{[4 punti]}
  La classe astratta \texttt{Time} rappresenta un istante della giornata,
  tramite il numero di secondi passati
  dall'inizio della giornata. La si modifichi
  dove indicato con \texttt{TODO}.
\end{esercizio}

\begin{esercizio}~\textbf{[6 punti]}
  La sottoclasse \texttt{ItalianTime} di \texttt{Time} implementa
  un istante temporale con notazione italiana. La si modifichi dove
  indicato con \texttt{TODO}.
\end{esercizio}

\begin{esercizio}~\textbf{[10 punti]}
  La sottoclasse \texttt{AmericanTime} di \texttt{Time} implementa
  un istante temporale con notazione degli Stati Uniti. La si modifichi dove
  indicato con \texttt{TODO}.
\end{esercizio}

\begin{esercizio}~\textbf{[11 punti]}
  La classe \texttt{Event} rappresenta un evento della giornata, che ha
  un nome, un istante di inizio e una durata in minuti. La si modifichi dove
  indicato con \texttt{TODO}.
\end{esercizio}

\mbox{}\\

\emph{Se tutto \`e  corretto, l'esecuzione del \texttt{Main}
  (gi\`a completo, non lo si modifichi)
  stamper\`a quanto riportato nel file di testo \texttt{risultato\_main.txt}.
}

\end{document}
