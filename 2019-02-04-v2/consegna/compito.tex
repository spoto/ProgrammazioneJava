\documentclass[12pt]{article}
\usepackage[pdftex]{graphicx}
\usepackage{amsfonts}
\usepackage[italian]{babel}
\usepackage{graphicx}
\usepackage{color}
\usepackage{multirow,bigdelim}

\definecolor{grey}{rgb}{0.3,0.3,0.3}

\usepackage{listings, framed}
\lstset{
  language=Java,
  showstringspaces=false,
  columns=flexible,
  basicstyle={\small\ttfamily},
  frame=none,
  numbers=none,
  keywordstyle=\bfseries\color{grey},
  commentstyle=\itshape\color{red},
  identifierstyle=\color{black},
  stringstyle=\color{blue},
  numberstyle={\ttfamily},
%  breaklines=true,
  breakatwhitespace=true,
  tabsize=3,
  escapechar=|
}

%****************enlarge layout
\textheight     243.5mm
\topmargin      -20.0mm
\textwidth      480pt
\hoffset        -80pt
%*****************theorems and such
\newcounter{esnu}
\newenvironment{esercizio}{\medskip \noindent {\bf Esercizio\addtocounter{esnu}{1} \arabic{esnu}}}{}
\pagestyle{empty}
\newcommand{\liff}{\mathrel{\leftrightarrow}}   % Logical IFF Symbol
\newcommand{\metaiff}{\Longleftrightarrow}      %iff in metatheory

\begin{document}

%\begin{tabular}{llclcr}
% \hspace{-35pt} &{\bf COGNOME:} & \hspace{100pt}        &{\bf NOME:}    & \hspace{100pt}        &{\bf MATRICOLA:}%\hspace{35pt} \\
%\hline
%\end{tabular}
\begin{center} {\bf Esame di Programmazione II, 4 febbraio 2019, turno 2}\end{center}
%\`

\emph{
Si crei un progetto Eclipse e, nella directory dei sorgenti,
si crei il package \texttt{it.univr.agenda}. Si copi al suo interno
le classi del compito, tranne \texttt{Main.java} che va copiato
dentro il package di default.
Se si realizzano nuove classi, le si creino dentro
il package \texttt{it.univr.agenda}.
Non si modifichino le dichiarazioni dei metodi. Si possono definire altri campi,
metodi, costruttori e classi, ma devono essere \texttt{private}.
La consegna fornita compila.
Anche la soluzione che verr\`a consegnata dovr\`a compilare,
altrimenti non verr\`a corretta.
}

\mbox{}\\

Una data viene rappresentata in Italia
con giorno, mese e anno, separati da un carattere slash, come per esempio 04/02/2019.
Esiste anche la rappresentazione trimestrale di una data, rappresentata con il giorno
di un trimestre (in inglese, \emph{quarter})
di un anno. Per esempio, 35Q1 2019 \`e il 35-esimo giorno del primo
trimestre del 2019, cio\`e sempre lo 04/02/2019. I quattro trimestri di un anno
si indicano con Q1, Q2, Q3 e Q4.

Per esempio, alcune date
sono rappresentate in questo modo in Italia e con la rappresentazione trimestrale:
%
\begin{center}
  \begin{tabular}{|c|c|}
    \hline
    Italia & trimestrale \\\hline\hline
    04/02/2019 & 35Q1 2019 \\\hline
    01/03/2015 & 60Q1 2015 \\\hline
    01/04/2015 & 01Q2 2015 \\\hline
    01/03/2016 & 61Q1 2016 \\\hline
    01/04/2016 & 01Q2 2016 \\\hline
  \end{tabular}
\end{center}
%
Si noti che il 2016 \`e bisestile, quindi il suo primo marzo \`e il 61-esimo giorno del primo trimestre,
mentre il primo marzo del 2015 \`e il 60-esimo giorno del primo trimestre.

\begin{esercizio}~\textbf{[4 punti]}
  La classe astratta \texttt{Date} rappresenta una data di un anno dal 1900 in poi,
  tramite il numero di giorni passati
  dall'inizio del 1900. La si modifichi
  dove indicato con \texttt{TODO}.
\end{esercizio}

\begin{esercizio}~\textbf{[6 punti]}
  La sottoclasse \texttt{ItalianDate} di \texttt{Date} implementa
  una data italiana. La si modifichi dove
  indicato con \texttt{TODO}. Si noti che la classe contiene dei
  metodi privati, che dovrebbero risultarvi utili.
\end{esercizio}

\begin{esercizio}~\textbf{[10 punti]}
  La sottoclasse \texttt{QuarterDate} di \texttt{Date} implementa
  una data trimestrale. La si modifichi dove
  indicato con \texttt{TODO}. Si noti che la classe contiene dei
  metodi privati, che dovrebbero risultarvi utili.
\end{esercizio}

\begin{esercizio}~\textbf{[11 punti]}
  La classe \texttt{Event} rappresenta un evento, che ha
  un nome, una data di inizio e una durata in giorni. La si modifichi dove
  indicato con \texttt{TODO}.
\end{esercizio}

\mbox{}\\

\emph{Se tutto \`e  corretto, l'esecuzione del \texttt{Main}
  (gi\`a completo, non lo si modifichi)
  stamper\`a quanto riportato nel file di testo \texttt{risultato\_main.txt}.
}

\end{document}
