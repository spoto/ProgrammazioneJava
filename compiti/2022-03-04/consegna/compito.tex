\documentclass[12pt]{article}
\usepackage[pdftex]{graphicx}
\usepackage{amsfonts}
\usepackage[italian]{babel}
\usepackage{graphicx}
\usepackage{color}
\usepackage{multirow,bigdelim}
\usepackage{relsize}
\usepackage{fdsymbol}

\definecolor{grey}{rgb}{0.3,0.3,0.3}

\usepackage{listings, framed}
\lstset{
  language=Java,
  showstringspaces=false,
  columns=flexible,
  basicstyle={\small\ttfamily},
  frame=none,
  numbers=none,
  keywordstyle=\bfseries\color{grey},
  commentstyle=\itshape\color{red},
  identifierstyle=\color{black},
  stringstyle=\color{blue},
  numberstyle={\ttfamily},
%  breaklines=true,
  breakatwhitespace=true,
  tabsize=3,
  escapechar=|
}

\def\codesize{\smaller}
\def\<#1>{\codeid{#1}}
\newcommand{\codeid}[1]{\ifmmode{\mbox{\codesize\ttfamily{#1}}}\else{\codesize\ttfamily #1}\fi}

%****************enlarge layout
\textheight     243.5mm
\topmargin      -20.0mm
\textwidth      480pt
\hoffset        -80pt
%*****************theorems and such
\newcounter{esnu}
\newenvironment{esercizio}{\medskip \noindent {\bf Esercizio\addtocounter{esnu}{1} \arabic{esnu}}}{}
\pagestyle{empty}
\newcommand{\liff}{\mathrel{\leftrightarrow}}   % Logical IFF Symbol
\newcommand{\metaiff}{\Longleftrightarrow}      %iff in metatheory

\begin{document}

\begin{center} {\bf Esame di Programmazione II, 4 marzo 2022}\end{center}

\emph{
Si crei un progetto Eclipse e
il package \texttt{it.univr.supermarket}. Si copino al suo interno
le classi del compito.
Non si modifichino le dichiarazioni dei metodi. Si possono definire altri campi,
metodi, costruttori e classi, ma devono essere \texttt{private}.
La soluzione che verr\`a consegnata dovr\`a compilare,
altrimenti non verr\`a corretta.}

\vspace*{2ex}

\begin{center}{\Large Esercizio 1} ($9$ punti)\\
  \textbf{(si consegni \texttt{Product.java})}
\end{center}

Si completi la classe astratta \texttt{Product.java}. Essa
rappresenta un prodotto di un supermercato, con
un nome e un prezzo in euro.
Si noti che \<Product> \`e una classe astratta, quindi ci sono dei metodi che dovranno
essere implementati solo nelle sue sottoclassi.
Si noti che ci sono due metodi \<getPrice>: quello che restituisce il prezzo come
era fornito al costruttore e quello che restituisce il prezzo in un certo momento di tempo:
se fosse nelle 24 ore prima della scadenza, questo secondo prezzo \`e scontato del 40\%
rispetto al prezzo fornito al costruttore.
Tale momento di tempo \`e espresso
in millisecondi dall'1/1/1970, che \`e lo stesso formato ritornato da \<System.currentTimeMillis()>.

\vspace*{2ex}

\begin{center}{\Large Esercizio 2} ($9$ punti)\\
  \textbf{(si consegni \texttt{ProductNotExpiring.java} e \texttt{ProductWithExpiration.java})}
\end{center}
  
Si completino
le classi concrete \<ProductNotExpiring.java> e \<ProductWithExpiration.java>, che estendono
\<Product> e rappresentano, rispettivamente, prodotti che non scadono mai (e quindi non vengono
mai scontati) e prodotti che hanno
un momento di scadenza. Si leggano bene i commenti sopra i metodi per capire come implementarli.

\vspace*{2ex}

\begin{center}{\Large Esercizio 3} ($8$ punti)\\
  \textbf{(si consegni \texttt{Supermarket.java})}
\end{center}

Si completi la classe \<Supermarket.java>, che contiene dei prodotti.
Si noti che tale classe ha due metodi \<toString>: quello che riceve come parametro
il momento in cui descrivere
il supermercato e quello che descrive il supermercato al momento attuale (\<System.currentTimeMillis()>).
Si leggano bene i commenti dei metodi.

\vspace*{2ex}

\begin{center}{\Large Esercizio 4} ($5$ punti)\\
  \textbf{(si consegni \texttt{Main.java})}
\end{center}

Si completi la classe \<Main.java>.

\vspace*{2ex}
\hrule
\vspace*{2ex}

Se tutto \`e corretto, l'esecuzione del \<Main> deve stampare:

{\scriptsize\begin{verbatim}
melone marcio: expired
pane: 1.20 euros
pane: 1.00 euros
carote: 2.00 euros
cipolle: 1.70 euros
mozzarella: 4.50 euros
uova: 2.50 euros
acqua: 2.20 euros
bagnoschiuma: 4.90 euros
dentifricio: 2.98 euros

melone marcio: expired
pane: 0.72 euros (special offer)
pane: 1.00 euros
carote: 2.00 euros
cipolle: 1.70 euros
mozzarella: 4.50 euros
uova: 2.50 euros
acqua: 2.20 euros
bagnoschiuma: 4.90 euros
dentifricio: 2.98 euros

melone marcio: expired
pane: expired
pane: 0.60 euros (special offer)
carote: 2.00 euros
cipolle: 1.70 euros
mozzarella: 4.50 euros
uova: 2.50 euros
acqua: 2.20 euros
bagnoschiuma: 4.90 euros
dentifricio: 2.98 euros

melone marcio: expired
pane: expired
pane: expired
carote: 2.00 euros
cipolle: 1.70 euros
mozzarella: 4.50 euros
uova: 2.50 euros
acqua: 2.20 euros
bagnoschiuma: 4.90 euros
dentifricio: 2.98 euros

melone marcio: expired
pane: expired
pane: expired
carote: 1.20 euros (special offer)
cipolle: 1.02 euros (special offer)
mozzarella: 4.50 euros
uova: 2.50 euros
acqua: 2.20 euros
bagnoschiuma: 4.90 euros
dentifricio: 2.98 euros

melone marcio: expired
pane: expired
pane: expired
carote: expired
cipolle: expired
mozzarella: 2.70 euros (special offer)
uova: 2.50 euros
acqua: 2.20 euros
bagnoschiuma: 4.90 euros
dentifricio: 2.98 euros

melone marcio: expired
pane: expired
pane: expired
carote: expired
cipolle: expired
mozzarella: expired
uova: 2.50 euros
acqua: 2.20 euros
bagnoschiuma: 4.90 euros
dentifricio: 2.98 euros
\end{verbatim}}

\end{document}
