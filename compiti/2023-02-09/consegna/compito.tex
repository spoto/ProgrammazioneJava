\documentclass[12pt]{article}
\usepackage[pdftex]{graphicx}
\usepackage{amsfonts}
\usepackage[italian]{babel}
\usepackage{graphicx}
\usepackage{color}
\usepackage{multirow,bigdelim}
\usepackage{relsize}
\usepackage{fdsymbol}
\usepackage{mdframed}

\definecolor{grey}{rgb}{0.3,0.3,0.3}
\definecolor{verylightgray}{rgb}{.97,.97,.97}
\definecolor{lightred}{rgb}{.97,.50,.50}

\usepackage{listings, framed}
\lstset{
  language=Java,
  showstringspaces=false,
  columns=flexible,
  basicstyle={\small\ttfamily},
  frame=none,
  numbers=none,
  keywordstyle=\bfseries\color{grey},
  commentstyle=\itshape\color{red},
  identifierstyle=\color{black},
  stringstyle=\color{blue},
  numberstyle={\ttfamily},
%  breaklines=true,
  breakatwhitespace=true,
  tabsize=3,
  escapechar=|
}

\def\codesize{\smaller}
\def\<#1>{\codeid{#1}}
\newcommand{\codeid}[1]{\ifmmode{\mbox{\codesize\ttfamily{#1}}}\else{\codesize\ttfamily #1}\fi}

%****************enlarge layout
\textheight     243.5mm
\topmargin      -20.0mm
\textwidth      480pt
\hoffset        -80pt
%*****************theorems and such
\newcounter{esnu}
\newenvironment{esercizio}{\medskip \noindent {\bf Esercizio\addtocounter{esnu}{1} \arabic{esnu}}}{}
\pagestyle{empty}
\newcommand{\liff}{\mathrel{\leftrightarrow}}   % Logical IFF Symbol
\newcommand{\metaiff}{\Longleftrightarrow}      %iff in metatheory

\begin{document}

\begin{center} {\bf Esame di Programmazione II, 9 febbraio 2023}\end{center}

\emph{
Si crei un progetto Eclipse e
il package \<it.univr.doodle>. Si copino al suo interno
le classi del compito.
Non si modifichino le dichiarazioni dei metodi. Si possono definire altri campi,
metodi, costruttori e classi, ma devono essere \<private>.
La soluzione che verr\`a consegnata dovr\`a compilare,
altrimenti non verr\`a corretta.}

\vspace*{2ex}

Si vuole implementare un \emph{doodle}, cio\`e uno strumento con cui decidere
il momento in cui svolgere un evento, massimizzando la presenza dei partecipanti. Ogni partecipante \`e una persona che pu\`o essere disponibile
in alcuni slot temporali. Il doodle permette di vedere qual \`e lo slot
al quale sono disponibili pi\`u persone. Normalmente un doodle d\`a a tutte
le persone la stessa priorit\`a. Ma un doodle pesato (\emph{weighted}) d\`a
priorit\`a alle persone sulla base del loro ruolo (capo azienda, programmatore,
ecc.).

\vspace*{3ex}

\begin{center}{\Large Esercizio 1} ($7$ punti)\\
  \textbf{(si consegni \<Slot.java>)}
\end{center}

Uno slot temporale specifica un momento temporale in cui si potrebbe svolgere l'evento.
Si tratta di un oggetto comparabile, che deve essere inseribile in una collezione
ordinata ma anche in un insieme o mappa hash. Si completi l'implementazione fornita di \<Slot>.

\vspace*{2ex}

\begin{center}{\Large Esercizio 2} ($5$ punti)\\
  \textbf{(si consegni \<Person.java>)}
\end{center}

La classe astratta
\<Person> rappresenta una persona che pu\`o partecipare al doodle.
Si completi \<Person.java> in modo da rendere le persone comparabili fra di loro
(mettendole in ordine di priorit\`a crescente e, a parit\`a di priorit\`a,
in ordine alfabetico per nome). Nota: per questo esercizio dovrete
aggiungere un metodo pubblico. Quale?

\vspace*{2ex}

\begin{center}{\Large Esercizio 3} ($4$ punti)\\
  \textbf{(si consegni \<CEO.java>, \<CTO.java>, \<Programmer.java> e \<Secretary.java>)}
\end{center}

Si definiscano queste quattro sottoclassi concrete
di \<Person>, con un costruttore che
le costruisce specificando il nome (come in \<Person>). La priorit\`a
del CEO (capo azienda) \`e 4; quella del CTO (capo tecnico) \`e 3;
quella del programmatore \`e 2; quella del segretario \`e 1.

\vspace*{2ex}

\begin{center}{\Large Esercizio 4} ($10$ punti)\\
  \textbf{(si consegni \<Doodle.java>)}
\end{center}

La classe che implementa il doodle fornisce metodi per la specifica della disponibilit\`a temporale di
ciascun partecipante. Inoltre ha un metodo \texttt{toString} che ritorna una
stringa che descrive il doodle, sotto forma di una tabella
(si veda l'esempio in fondo al compito). Si noti che questa tabella
ha sulle ascisse gli slot temporali in ordine crescente e sulle
ordinate i partecipanti in ordine crescente. Si completi l'implementazione
fornita di \<Doodle.java>. Si noti che questa classe d\`a a tutti i
partecipanti la stessa priorit\`a (1).

\vspace*{2ex}

\begin{center}{\Large Esercizio 5} ($5$ punti)\\
  \textbf{(si consegni \<WeightedDoodle.java>)}
\end{center}

Si definisca una sottoclasse \<WeightedDoodle> di \<Doodle> la cui unica
differenza \`e che essa usa le priorit\`a delle persone per pesare
gli slot temporali con maggior partecipazione. Questo significa per esempio che
uno slot a cui pu\`o partecipare il CEO e un segretario vale 5 (4+1)
mentre uno slot a cui possono partecipare due programmatori vale 4 (2+2).

\vspace*{3ex}
\hrule
\vspace*{2ex}

Se tutto \`e corretto, l'esecuzione del \<Main.java> stamper\`a
qualcosa del tipo:

\begin{mdframed}[backgroundcolor=lightred]
  {\small\begin{verbatim}
doodle1:
4/2/2017 MORNING  4/2/2017 AFTERNOON  5/2/2017 AFTERNOON  5/2/2017 EVENING  
  yes               no                  yes                 no    Alessandro
  no                no                  yes                 no    Alessandra
  yes               no                  yes                 yes   Giovanni
  no                yes                 no                  yes   Fausto
  2                 1                   3*                  2    

doodle2:
4/2/2017 MORNING  4/2/2017 AFTERNOON  5/2/2017 AFTERNOON  5/2/2017 EVENING  
  yes               no                  yes                 no    Alessandro
  no                no                  yes                 no    Alessandra
  yes               no                  yes                 yes   Giovanni
  no                yes                 no                  yes   Fausto
  4                 4                   6                   7*
\end{verbatim}}
\end{mdframed}

\noindent
Il primo \`e un doodle normale, che d\`a a tutti la priorit\`a 1 e quindi
seleziona il 5/2/2017 AFTERNOON come momento migliore (si noti l'asterico).
Il secondo \`e un doodle pesato, che usa la priorit\`a delle persone
data dal loro ruolo e seleziona il 5/2/2017 EVENING come momento migliore
(si noti l'asterisco).

\end{document}
