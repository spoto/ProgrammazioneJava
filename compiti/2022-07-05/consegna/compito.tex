\documentclass[12pt]{article}
\usepackage[pdftex]{graphicx}
\usepackage{amsfonts}
\usepackage[italian]{babel}
\usepackage{graphicx}
\usepackage{color}
\usepackage{multirow,bigdelim}
\usepackage{relsize}
\usepackage{fdsymbol}

\definecolor{grey}{rgb}{0.3,0.3,0.3}

\usepackage{listings, framed}
\lstset{
  language=Java,
  showstringspaces=false,
  columns=flexible,
  basicstyle={\small\ttfamily},
  frame=none,
  numbers=none,
  keywordstyle=\bfseries\color{grey},
  commentstyle=\itshape\color{red},
  identifierstyle=\color{black},
  stringstyle=\color{blue},
  numberstyle={\ttfamily},
%  breaklines=true,
  breakatwhitespace=true,
  tabsize=3,
  escapechar=|
}

\def\codesize{\smaller}
\def\<#1>{\codeid{#1}}
\newcommand{\codeid}[1]{\ifmmode{\mbox{\codesize\ttfamily{#1}}}\else{\codesize\ttfamily #1}\fi}

%****************enlarge layout
\textheight     243.5mm
\topmargin      -20.0mm
\textwidth      480pt
\hoffset        -80pt
%*****************theorems and such
\newcounter{esnu}
\newenvironment{esercizio}{\medskip \noindent {\bf Esercizio\addtocounter{esnu}{1} \arabic{esnu}}}{}
\pagestyle{empty}
\newcommand{\liff}{\mathrel{\leftrightarrow}}   % Logical IFF Symbol
\newcommand{\metaiff}{\Longleftrightarrow}      %iff in metatheory

\begin{document}

\begin{center} {\bf Esame di Programmazione II, 5 luglio 2022}\end{center}

\emph{
Si crei un progetto Eclipse e
il package \<it.univr.elezioni>. Si copino al suo interno
le classi del compito.
Non si modifichino le dichiarazioni dei metodi. Si possono definire altri campi,
metodi, costruttori e classi, ma devono essere \<private>.
La soluzione che verr\`a consegnata dovr\`a compilare,
altrimenti non verr\`a corretta.}

\vspace*{2ex}

\begin{center}{\Large Esercizio 1} ($6$ punti)\\
  \textbf{(si consegni \<Partito.java>)}
\end{center}

Si completi la classe \<Partito.java>, che rappresenta un partito politico.
Un partito ha un nome. Comparando due partiti, viene prima quello con il
nome pi\`u piccolo in ordine alfabetico.

\vspace*{2ex}

\begin{center}{\Large Esercizio 2} ($18$ punti)\\
  \textbf{(si consegni \<Elezioni.java>)}
\end{center}

Si completi la classe \<Elezioni.java>, che rappresenta delle elezioni.
Gli oggetti di tale classe permettono di registrare voti per dei partiti.
Il metodo \<toString()> restituisce una stringa che descrive i risultati
dell'elezione, del tipo:
\begin{verbatim}
1        Bassotti:  4467 voti (28.11%)
2         Caotico:  4679 voti (29.45%)
3          Felice:  1591 voti (10.01%)
4        Floreale:  3950 voti (24.86%)
5      Pensionati:  1202 voti (7.56%)
\end{verbatim}
Si noti che i partiti devono essere riportati in ordine crescente,
con un indice progressivo alla sinistra e con l'indicazione dei voti ottenuti
alla destra, pi\`u la percentuale dei voti ottenuti da ciascun partito,
con due cifre di precisione dopo il punto decimale.

\vspace*{2ex}

\begin{center}{\Large Esercizio 3} ($7$ punti)\\
  \textbf{(si consegni \<ElezioniVincitore.java>)}
\end{center}
  
Si completi la sottoclasse \<ElezioniVincitore.java> di
\<Elezione.java>, la cui unica differenza rispetto alla superclasse
\`e che il metodo \<toString()> aggiunge in fondo l'indicazione
del partito che ha vinto, del tipo:
\begin{verbatim}
1        Bassotti:  4467 voti (28.11%)
2         Caotico:  4679 voti (29.45%)
3          Felice:  1591 voti (10.01%)
4        Floreale:  3950 voti (24.86%)
5      Pensionati:  1202 voti (7.56%)

Vince Caotico
\end{verbatim}
A parit\`a di voti, vince il partito che viene prima in ordine alfabetico.
Se non ci fossero partiti (elezioni vuote) va aggiunta invece
la stringa \<Non ci sono vincitori>.

\vspace*{2ex}
\hrule
\vspace*{2ex}

Se tutto \`e corretto, l'esecuzione del \<Main.java> gi\`a scritto stamper\`a
un'elezione casuale con l'indicazione del vincitore in basso.

\end{document}
