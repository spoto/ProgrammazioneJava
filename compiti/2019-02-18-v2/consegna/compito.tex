\documentclass[12pt]{article}
\usepackage[pdftex]{graphicx}
\usepackage{amsfonts}
\usepackage[italian]{babel}
\usepackage{graphicx}
\usepackage{color}
\usepackage{multirow,bigdelim}

\definecolor{grey}{rgb}{0.3,0.3,0.3}

\usepackage{listings, framed}
\lstset{
  language=Java,
  showstringspaces=false,
  columns=flexible,
  basicstyle={\small\ttfamily},
  frame=none,
  numbers=none,
  keywordstyle=\bfseries\color{grey},
  commentstyle=\itshape\color{red},
  identifierstyle=\color{black},
  stringstyle=\color{blue},
  numberstyle={\ttfamily},
%  breaklines=true,
  breakatwhitespace=true,
  tabsize=3,
  escapechar=|
}

%****************enlarge layout
\textheight     243.5mm
\topmargin      -20.0mm
\textwidth      480pt
\hoffset        -80pt
%*****************theorems and such
\newcounter{esnu}
\newenvironment{esercizio}{\medskip \noindent {\bf Esercizio\addtocounter{esnu}{1} \arabic{esnu}}}{}
\pagestyle{empty}
\newcommand{\liff}{\mathrel{\leftrightarrow}}   % Logical IFF Symbol
\newcommand{\metaiff}{\Longleftrightarrow}      %iff in metatheory

\begin{document}

%\begin{tabular}{llclcr}
% \hspace{-35pt} &{\bf COGNOME:} & \hspace{100pt}        &{\bf NOME:}    & \hspace{100pt}        &{\bf MATRICOLA:}%\hspace{35pt} \\
%\hline
%\end{tabular}
\begin{center} {\bf Esame di Programmazione II, 18 febbraio 2019}\end{center}
%\`

\emph{
Si crei un progetto Eclipse e, nella directory dei sorgenti,
si crei il package \texttt{it.univr.library}. Si copino al suo interno
le classi del compito, tranne \texttt{Main.java} che va copiato
dentro il package di default.
Se si realizzano nuove classi, le si creino dentro
il package \texttt{it.univr.library}.
Non si modifichino le dichiarazioni dei metodi. Si possono definire altri campi,
metodi, costruttori e classi, ma devono essere \texttt{private}.
La consegna fornita compila a meno di due costruttori che non
compilano.
La soluzione che verr\`a consegnata dovr\`a compilare,
altrimenti non verr\`a corretta.
}

\mbox{}\\

Lo scopo del compito \`e di realizzare un catalogo dei libri di una
biblioteca. I libri possono essere cartacei
(\texttt{PaperBook}) oppure audio (\texttt{AudioBook}).
Un catalogo contiene vari libri, senza ripetizioni, e permette di specificare
l'ordinamento con cui vanno tenuti dentro al catalogo stesso.
I generi dei libri (storia, informatica, narrativa\ldots)
sono enumerati dentro \texttt{Genre}, che \`e gi\`a completa
e non deve venire modificata.

\begin{esercizio}~\textbf{[4 punti]}
  La classe astratta \texttt{Book} rappresenta un libro, genericamente,
  senza ancora distinguere fra libri cartacei e audio.
  La si completi dove indicato con \texttt{TODO}.
\end{esercizio}

\begin{esercizio}~\textbf{[6 punti]}
  Le sottoclassi concrete \texttt{PaperBook} e \texttt{AudioBook}
  di \texttt{Book} implementano i due diversi tipi di libro.
  Le si completino dove indicato con \texttt{TODO}.
\end{esercizio}

\begin{esercizio}~\textbf{[7 punti]}
  La classe \texttt{Catalog} implementa il catalogo dei libri di
  una biblioteca. Quindi la sua costruzione richiede di specificare i
  libri che devono essere contenuti dentro il catalogo. Tali libri
  sono tenuti ordinati. Questo significa che stampando un catalogo si
  vedranno i libri in ordine crescente e che
  iterando su un catalogo si otterrano i
  libri in ordine crescente. L'ordine dei libri non \`e necessariamente
  quello del \texttt{compareTo} fra i libri: \`e possibile infatti
  specificare un ordinamento diverso al momento della costruzione del
  catalogo, passando un oggetto di tipo \texttt{java.util.Comparator}.
  Si completi la classe dove indicato con \texttt{TODO}.
\end{esercizio}

\begin{esercizio}~\textbf{[5 punti]}
  La classe \texttt{CatalogWithStatistics} si comporta come la sua
  superclasse \texttt{Catalog} ma ha un \texttt{toString} che
  aggiunge, dopo la lista dei libri, una riga che riporta il numero totale
  di pagine del libri cartacei e il numero totale di minuti degli audio-libri.
  Si completi la classe dove indicato con \texttt{TODO}.
\end{esercizio}

\begin{esercizio}~\textbf{[9 punti]}
  Si completi la classe di partenza \texttt{Main}, dove indicato con
  \texttt{TODO}. In particolare, si dovr\`a creare tre cataloghi e stamparli.
  Tutti e tre i cataloghi contengono gli stessi libri, ma usano tre
  ordinamenti distinti. Inoltre il primo riporta anche il numero di pagine
  e minuti totali dei libri.
\end{esercizio}

\mbox{}\\

\emph{Se tutto \`e  corretto, l'esecuzione del \texttt{Main}
  (gi\`a completo, non lo si modifichi)
  stamper\`a quanto riportato nel file di testo \texttt{risultato\_main.txt}.
}

\end{document}
