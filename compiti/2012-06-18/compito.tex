\documentclass{article}[10pt]
\usepackage[pdftex]{graphicx}
\usepackage{amsfonts}
\usepackage[italian]{babel}
%****************enlarge layout
\textheight     243.5mm
\topmargin      -20.0mm
\textwidth      480pt
\hoffset        -80pt
%*****************theorems and such
\newcounter{esnu}
\newenvironment{esercizio}{\medskip \noindent {\bf Esercizio\addtocounter{esnu}{1} \arabic{esnu}}}{}
\pagestyle{empty}
\newcommand{\liff}{\mathrel{\leftrightarrow}}   % Logical IFF Symbol
\newcommand{\metaiff}{\Longleftrightarrow}      %iff in metatheory

\begin{document}

%\begin{tabular}{llclcr}
% \hspace{-35pt} &{\bf COGNOME:} & \hspace{100pt}        &{\bf NOME:}    & \hspace{100pt}        &{\bf MATRICOLA:}%\hspace{35pt} \\
%\hline
%\end{tabular}
\begin{center} {\bf Esame di Programmazione II, 18 giugno 2012}\end{center}

Si consideri il seguente programma:
%
{\small
\begin{verbatim}
import java.util.HashSet;
import java.util.Set;

public class Main {

  public static void main(String[] args) {
    Polynomial poly1 = new Polynomial(new int[] { -3, 4, 0, 0, -2} );
    Polynomial poly2 = new Polynomial(new int[] { 5, -11, 0, 0 } );
    Polynomial poly3 = new Polynomial(new int[] { 3, -4, 3, 1, -1} );
    Polynomial poly4 = new Polynomial(new int[] { 0, 0, -3, 4, 0, 0, -2} );
    Polynomial poly5 = new Polynomial(new int[] { 4, 2, -1 } );

    System.out.println("poly1: "+ poly1 + " with degree " + poly1.degree());
    System.out.println("poly2: "+ poly2 + " with degree " + poly2.degree());
    System.out.println("poly3: "+ poly3 + " with degree " + poly3.degree());
    System.out.println("poly4: "+ poly4 + " with degree " + poly4.degree());
    System.out.println("poly5: "+ poly5 + " with degree " + poly5.degree());

    Polynomial sum = poly1.add(poly2);
    System.out.println("poly1 + poly2: " + sum + " with degree " + sum.degree());

    sum = poly1.add(poly3);
    System.out.println("poly1 + poly3: " + sum + " with degree " + sum.degree());

    SecondDegreePolynomial poly6 = new SecondDegreePolynomial(4, 2, -1);
    System.out.println(poly6 + " can be 0: " + poly6.canBe0());

    SecondDegreePolynomial poly7 = new SecondDegreePolynomial(4, 2, 3);
    System.out.println(poly7 + " can be 0: " + poly7.canBe0());
    System.out.println(poly7 + " at x = 7 is " + poly7.evaluate(7));

    Set<Polynomial> set = new HashSet<Polynomial>();
    set.add(poly1);
    set.add(poly2);
    set.add(poly3);
    set.add(poly4);
    set.add(poly5);
    set.add(poly6);
    set.add(poly7);

    System.out.println("set: " + set);

    // questo e' illegale:
    new SecondDegreePolynomial(0, 2, 3);
  }
}
\end{verbatim}
}

\noindent
che stampa:
%
{\small
\begin{verbatim}
poly1:  - 3x^4 + 4x^3 - 2x^0 with degree 4
poly2:  + 5x^3 - 11x^2 with degree 3
poly3:  + 3x^4 - 4x^3 + 3x^2 + 1x^1 - 1x^0 with degree 4
poly4:  - 3x^4 + 4x^3 - 2x^0 with degree 4
poly5:  + 4x^2 + 2x^1 - 1x^0 with degree 2
poly1 + poly2:  - 3x^4 + 9x^3 - 11x^2 - 2x^0 with degree 4
poly1 + poly3:  + 3x^2 + 1x^1 - 3x^0 with degree 2
 + 4x^2 + 2x^1 - 1x^0 can be 0: true
 + 4x^2 + 2x^1 + 3x^0 can be 0: false
 + 4x^2 + 2x^1 + 3x^0 at x = 7 is 213
set: [ ................. ]
Exception in thread "main" java.lang.IllegalArgumentException: first coefficient cannot be 0
\end{verbatim}
}

\begin{esercizio}
\textbf{[14 punti]}
Un \texttt{Polynomial} rappresenta un polinomio a coefficienti interi:
\[
  c_nx^n+c_{n-1}x^{n-1}+\ldots+c_1x+c_0
\]
dove $c_n\cdots c_0$ sono i coefficienti del polinomio, in $\mathbb{Z}$.
Il \emph{grado} (\emph{degree}) del polinomio \`e il pi\'u grande $d$ tale che $c_d\not=0$.

La classe \texttt{Polynomial} deve mettere a disposizione il costruttore pubblico:
%
\begin{itemize}
\item \texttt{Polynomial(int[] coefficients)}, che costruisce il polinomio con i coefficienti forniti
      (il primo elemento dell'array \`e $c_n$, l'ultimo \`e $c_0$).
\end{itemize}
%
Inoltre deve mettere a disposizione il metodo protetto:
%
\begin{itemize}
\item \texttt{int[] getCoefficients()}, che restituisce i coefficienti del polinomio;
\end{itemize}
%
e i metodi pubblici:
%
\begin{itemize}
\item \texttt{int degree()}, che restituisce il grado del polinomio;
\item \texttt{Polynomial add(Polynomial other)}, che restituisce un polinomio ottenuto sommando
      \texttt{other} a \texttt{this};
\item \texttt{int evaluate(int x)}, che restituisce il valore del polinomio al punto indicato;
\item \texttt{String toString()}, che restituisce una stringa che descrive il polinomio
      (si vedano gli esempi alla pagina precedente).
      In particolare, non deve riportare i monomi di coefficiente 0;
\item \texttt{boolean equals(Object other)}, che determina se \texttt{other} \`e un polinomio uguale
      a \texttt{this}; due polinomi sono uguali se e solo se hanno gli stessi coefficienti, senza considerare
      i coefficienti iniziali pari a 0 (nell'esempio della pagina precedente,
      \texttt{poly1} e \texttt{poly4} sono uguali);
\item \texttt{int hashCode()}, che restituisce un codice hash non banale per \texttt{this}. Per non banale
      si intende che non deve restituire sempre la stessa costante.
\end{itemize}
%
\end{esercizio}

\begin{esercizio}
\textbf{[7 punti]}
Si definisca la sottoclasse \texttt{SecondDegreePolynomial} di \texttt{Polynomial} che rappresenta
un polinomio di grado due: $ax^2+bx+c$. Tale classe deve mettere a disposizione il costruttore pubblico:
%
\begin{itemize}
\item \texttt{SecondDegreePolynomial(int a, int  b, int c)},
      che costruisce il polinomio di secondo grado con i coefficienti forniti,
      ma lancia una \texttt{java.lang.IllegalArgumentException} se $a$ \`e $0$.
\end{itemize}
%
Tutti i metodi di \texttt{Polynomial} devono essere applicabili anche a un \texttt{SecondDegreePolynomial} e
funzionare correttamente (serve ridefinirli esplicitamente?). Inoltre deve essere fornito un metodo
pubblico ulteriore:
%
\begin{itemize}
\item \texttt{boolean canBe0()}, che determina se il polinomio di secondo grado si pu\`o annullare
      (cio\`e se esiste un $x$ per cui $ax^2+bx+c = 0$).
\end{itemize}
%
\end{esercizio}
%
\begin{esercizio}
\textbf{[1 punto]}
Quanti polinomi vengono stampati dentro l'insieme \texttt{set} al posto dei puntini alla fine della pagina precedente,
in basso?
\end{esercizio}

\end{document}
