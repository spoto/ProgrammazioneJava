\documentclass[12pt]{article}
\usepackage[pdftex]{graphicx}
\usepackage{amsfonts}
\usepackage[italian]{babel}
\usepackage{graphicx}
\usepackage{color}
\usepackage{multirow,bigdelim}
\usepackage{relsize}
\usepackage{fdsymbol}
\usepackage{mdframed}

\definecolor{grey}{rgb}{0.3,0.3,0.3}
\definecolor{verylightgray}{rgb}{.97,.97,.97}
\definecolor{lightred}{rgb}{.97,.50,.50}

\usepackage{listings, framed}
\lstset{
  language=Java,
  showstringspaces=false,
  columns=flexible,
  basicstyle={\small\ttfamily},
  frame=none,
  numbers=none,
  keywordstyle=\bfseries\color{grey},
  commentstyle=\itshape\color{red},
  identifierstyle=\color{black},
  stringstyle=\color{blue},
  numberstyle={\ttfamily},
%  breaklines=true,
  breakatwhitespace=true,
  tabsize=3,
  escapechar=|
}

\def\codesize{\smaller}
\def\<#1>{\codeid{#1}}
\newcommand{\codeid}[1]{\ifmmode{\mbox{\codesize\ttfamily{#1}}}\else{\codesize\ttfamily #1}\fi}

%****************enlarge layout
\textheight     243.5mm
\topmargin      -20.0mm
\textwidth      480pt
\hoffset        -80pt
%*****************theorems and such
\newcounter{esnu}
\newenvironment{esercizio}{\medskip \noindent {\bf Esercizio\addtocounter{esnu}{1} \arabic{esnu}}}{}
\pagestyle{empty}
\newcommand{\liff}{\mathrel{\leftrightarrow}}   % Logical IFF Symbol
\newcommand{\metaiff}{\Longleftrightarrow}      %iff in metatheory

\begin{document}

\begin{center} {\bf Esame di Programmazione II, 24 febbraio 2023}\end{center}

\emph{
Si crei un progetto Eclipse e
il package \<it.univr.quindici>. Si copino al suo interno
le classi del compito.
Non si modifichino le dichiarazioni dei metodi e delle classi. Si possono definire altri campi,
metodi, costruttori e classi, ma devono essere \<private>.
La soluzione che verr\`a consegnata dovr\`a compilare,
altrimenti non verr\`a corretta.}

\vspace*{2ex}

Il gioco del 15 consiste in una matrice quadrata di 4x4 tessere di cui una sola \`e
vuota e le altre 15 sono riempite con i numeri tra 1 e 15. Il gioco \`e risolto se
la casella vuota \`e in basso a destra e
le tessere sono ordinate in senso crescente secondo una lettura per righe,
cio\`e se la matrice \`e nella configurazione:
%
\[\begin{array}{cccc}
 1 & 2 & 3 & 4\\
 5 & 6 & 7 & 8\\
 9 & 10 & 11 & 12\\
 13 & 14 & 15 & \\
\end{array}\]

Si intende adesso generalizzare questo gioco a una matrice di dimensioni generiche
$\mathit{width}\times\mathit{height}$ ($\mathit{width}$ righe e
$\mathit{height}$ colonne). Le tessere non sono
pi\`u solo numeriche, ma possono essere degli oggetti di tipo \<T>
con una relazione di ordinamento fra di loro. Ad esempio, un gioco
$4\times 3$ con tessere alfabetiche di lunghezza fra 1 e 5 \`e il seguente:
%
{\small
\begin{verbatim}
 swhv    kt     g  wohp 
sqvtc       nzwuo   evs 
  hkf    qb    lt    me 
\end{verbatim}
}

\noindent
Si noti che tale gioco non \`e risolto poich\'e le stringhe non sono in ordine
alfabetico secondo una lettura per righe, n\'e la casella vuota \`e in basso a destra.

\vspace*{3ex}

\begin{center}{\Large Esercizio 1} ($7$ punti)\\
  \textbf{(si consegni \<Tessera.java>)}
\end{center}

Una tessera \`e rappresentata dalla classe generica \<Tessera$\langle$T$\rangle$>, dove \<T>
\`e il tipo del valore contenuto nella tessera (una stringa, un intero, o altro tipo comparabile).
La classe, per come la trovate gi\`a scritta, vincola \<T> a essere comparabile con se stesso, in modo
che le tessere si possano mettere in ordine rispetto al valore contenuto:

\begin{mdframed}[backgroundcolor=lightred]
{\small\begin{verbatim}
class Tessera<T extends Comparable<T>> implements Comparable<Tessera<T>>
\end{verbatim}}
\end{mdframed}

\noindent
Si completi la classe \<Tessera$\langle$T$\rangle$> (metodi \<hashCode>,
\<toString> e \<compareTo>).
\vspace*{2ex}

\begin{center}{\Large Esercizio 2} ($5$ punti)\\
  \textbf{(si consegni \<FattoriaDiTessereNumeriche.java>)}
\end{center}

La classe astratta e generica
\<FattoriaDiTessere$\langle$T$\rangle$> rappresenta oggetti che generano
tessere di tipo \<Tessera$\langle$T$\rangle$>, casuali, ogni volta che si chiama
il metodo \<get> della fattoria. Tale classe (gi\`a completa)
estende la classe di libreria \<java.util.function.Supplier>.
Si definisca una sottoclasse \<FattoriaDiTessereNumeriche> di
\<FattoriaDiTessere$\langle$Integer$\rangle$> che definisce \<get> in modo
che a ogni chiamata si ottenga una \<Tessera$\langle$Integer$\rangle$>
casuale, con valore numerico contenuto tra 1 e \<max> inclusi, dove
\<max> viene specificato al costruttore di \<FattoriaDiTessereNumeriche>
(si veda il \<Main.java> per un esempio di utilizzo).

\vspace*{2ex}

\begin{center}{\Large Esercizio 3} ($6$ punti)\\
  \textbf{(si consegni \<FattoriaDiTessereAlfabetiche.java>)}
\end{center}

Si definisca una sottoclasse \<FattoriaDiTessereAlfabetiche> di
\<FattoriaDiTessere$\langle$String$\rangle$> che definisce \<get> in modo
che a ogni chiamata si ottenga una \<Tessera$\langle$String$\rangle$>
casuale, con valore stringa fatto da 1 a 5 lettere alfabetiche minuscole
(si veda il \<Main.java> per un esempio di utilizzo).

\vspace*{2ex}

\begin{center}{\Large Esercizio 4} ($13$ punti)\\
  \textbf{(si consegni \<Gioco.java>)}
\end{center}

Si completi la classe \<Gioco.java> che implementa un gioco
di dimensione \<width> x \<height> (\<width> colonne, \<height> righe).
Il suo costruttore (che dovete completare) crea il gioco casualmente (tessere casuali
\emph{tutte distinte}, casella vuota posizionata casualmente).
Il costruttore sa il modo per costruire tessere casuali poich\'e gli viene
passata una fattoria:

\begin{mdframed}[backgroundcolor=lightred]
\begin{verbatim}
public Gioco(FattoriaDiTessere<T> fattoria, int width, int height)
\end{verbatim}
\end{mdframed}

\noindent
e pu\`o quindi chiamarne il metodo \<get> ripetutamente per ottenere tessere casuali
(dovr\`a per\`o in qualche modo garantire che risultino tutte distinte).
Inoltre dovrete completare il metodo \<risolto> che determina se il gioco \`e risolto
(casella vuota in basso a destra, tessere ordinate crescenti in una lettura per righe).

\vspace*{2ex}
\hrule\mbox{}\\
\vspace*{2ex}

Se tutto \`e corretto, l'esecuzione del \<Main.java> (gi\`a fatto, da non modificare)
stamper\`a qualcosa del tipo:

\begin{mdframed}[backgroundcolor=lightred]
{\small\begin{verbatim}
 ttno         lzp    qy 
    h   zte    ta jigbz 
  zve     k xhjmf    at 
galtp    mj    yo    pd 

    1     8     3 
          5     2 

          8     4 
    1     2     7 

    // molti tentativi qui omessi per brevità, fino a un gioco risolto (sotto)

    1     2     3 
    5     7  
\end{verbatim}}
\end{mdframed}

\noindent
Come si vede dalla stampa (e dal codice), il \<Main.java> prima genera un gioco casuale
4x4 con tessere alfabetiche, poi genera dei giochi casuali 3x2 con tessere numeriche
tra 1 e 8, finch\'e non ottiene un gioco risolto e a quel punto termina.

\end{document}
