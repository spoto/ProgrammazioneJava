\documentclass[12pt]{article}
\usepackage[pdftex]{graphicx}
\usepackage{amsfonts}
\usepackage[italian]{babel}
\usepackage{graphicx}
\usepackage{color}
\usepackage{multirow,bigdelim}
\usepackage{relsize}
\usepackage{fdsymbol}
\usepackage{mdframed}

\definecolor{grey}{rgb}{0.3,0.3,0.3}
\definecolor{verylightgray}{rgb}{.97,.97,.97}
\definecolor{lightred}{rgb}{.97,.50,.50}

\usepackage{listings, framed}
\lstset{
  language=Java,
  showstringspaces=false,
  columns=flexible,
  basicstyle={\small\ttfamily},
  frame=none,
  numbers=none,
  keywordstyle=\bfseries\color{grey},
  commentstyle=\itshape\color{red},
  identifierstyle=\color{black},
  stringstyle=\color{blue},
  numberstyle={\ttfamily},
%  breaklines=true,
  breakatwhitespace=true,
  tabsize=3,
  escapechar=|
}

\def\codesize{\smaller}
\def\<#1>{\codeid{#1}}
\newcommand{\codeid}[1]{\ifmmode{\mbox{\codesize\ttfamily{#1}}}\else{\codesize\ttfamily #1}\fi}

%****************enlarge layout
\textheight     243.5mm
\topmargin      -20.0mm
\textwidth      480pt
\hoffset        -80pt
%*****************theorems and such
\newcounter{esnu}
\newenvironment{esercizio}{\medskip \noindent {\bf Esercizio\addtocounter{esnu}{1} \arabic{esnu}}}{}
\pagestyle{empty}
\newcommand{\liff}{\mathrel{\leftrightarrow}}   % Logical IFF Symbol
\newcommand{\metaiff}{\Longleftrightarrow}      %iff in metatheory

\begin{document}

\begin{center} {\bf Esame di Programmazione II, 2 settembre 2022}\end{center}

\emph{
Si crei un progetto Eclipse e
il package \<it.univr.letters>. Si copino al suo interno
le classi del compito.
Non si modifichino le dichiarazioni dei metodi. Si possono definire altri campi,
metodi, costruttori e classi, ma devono essere \<private>.
La soluzione che verr\`a consegnata dovr\`a compilare,
altrimenti non verr\`a corretta.}

\vspace*{2ex}

L'interfaccia \texttt{Letters} \`e gi\`a completa e rappresenta una sequenza
di lettere (cio\`e caratteri alfabetici):

\begin{mdframed}[backgroundcolor=verylightgray]
  \begin{lstlisting}[language=Java]
public interface Letters {
  int length(); // la lunghezza della sequenza
  String toString(); // la concatenazione delle sue lettere
  void forEach(Consumer<Character> command); // applica command alle sue lettere
}
  \end{lstlisting}
\end{mdframed}
%
Si noti che il metodo \<forEach> esegue del codice per
ogni lettera della sequenza (dalla prima all'ultima della sequenza),
e che questo codice \`e fornito a \<forEach> tramite un parametro di tipo
\<java.util.function.Consumer>, un'interfaccia della libreria standard
definita come segue:
%
\begin{mdframed}[backgroundcolor=verylightgray]
  \begin{lstlisting}[language=Java]
public interface Consumer<T> {
  void accept(T t); // fa qualcosa con t (cioe' lo "consuma")
}
  \end{lstlisting}
\end{mdframed}

\vspace*{3ex}

\begin{center}{\Large Esercizio 1} ($12$ punti)\\
  \textbf{(si consegni \<LowerCase.java>)}
\end{center}

Si completi la classe \<LowerCase.java>, un'implementazione di
\<Letters> che rappresenta una sequenza \emph{minuscola}, cio\`e
fatta da lettere minuscole dell'alfabeto inglese (sono ammesse ripetizioni).

\vspace*{2ex}

\begin{center}{\Large Esercizio 2} ($13$ punti)\\
  \textbf{(si consegni \<Vulcanian.java>)}
\end{center}

Si completi la classe \<Vulcanian.java>, una sottoclasse di
\<LowerCase> che rappresenta una sequenza \emph{vulcaniana},
cio\`e un caso particolare di sequenza minuscola
che \`e fatta da due parti: la prima parte contiene vocali
in ordine alfabetico;
la seconda parte contiene consonanti in ordine alfabetico.
Per esempio, una sequenza vulcaniana \`e \<iobcchjjknnnnnprsstx>:
\[
\underbrace{\mbox{\texttt{io}}}_{\text{vocali in ordine alfabetico}}\ \underbrace{\mbox{\texttt{bcchjjknnnnnprsstx}}}_{\text{consonanti in ordine alfabetico}}
\]
Invece \<ioBcchjjknnnnnprsstx> non \`e vulcaniana perch\'e non \`e
neppure minuscola e per lo stesso motivo non lo \`e
\<io(cchjjknnnnnprsstx>.
Neanche \<ibocchjjknnnnnprsstx> \`e vulcaniana perch\'e
la vocale \<o> segue la consonante \<b>.
Neanche \<oibcchjjknnnnnprsstx> \`e vulcaniana perch\'e
le vocali non sono in ordine alfabetico.
E infine neanche \<iocbchjjknnnnnprsstx> \`e vulcaniana
perch\'e le consonanti non sono in ordine alfabetico.

\vspace*{2ex}

\begin{center}{\Large Esercizio 3} ($6$ punti)\\
  \textbf{(si consegni \<Main.java>)}
\end{center}

Si completi la classe \<Main.java>.

\vspace*{3ex}
\vspace*{2ex}
\hrule
\vspace*{2ex}

Se tutto \`e corretto, l'esecuzione del \<Main.java> stamper\`a
qualcosa del tipo:

\begin{mdframed}[backgroundcolor=lightred]
  \begin{verbatim}
l1 = othaj
I suoi caratteri sono:
o
t
h
a
j

l2 = eeeuccfhhjjknqrvvwxy
I suoi caratteri non italiani sono:
j
j
k
w
x
y

new LowerCase("tmwdmghbonqlwbwpiqkv") => successo
new LowerCase("tmwd(ghbonqlwbwpiqkv") => eccezione
new LowerCase("tmwDmghbonqlwbwpiqkv") => eccezione
new Vulcanian("iobcchjjknnnnnprsstx") => successo
new Vulcanian("ioBcchjjknnnnnprsstx") => eccezione
new Vulcanian("io(cchjjknnnnnprsstx") => eccezione
new Vulcanian("ibocchjjknnnnnprsstx") => eccezione
new Vulcanian("oibcchjjknnnnnprsstx") => eccezione
new Vulcanian("iocbchjjknnnnnprsstx") => eccezione
\end{verbatim}
\end{mdframed}

\end{document}
